\documentclass[twoside]{report}
\usepackage{supertab}

\makeindex
\leftmargin 0in
\oddsidemargin 0in
\evensidemargin 0in
\topmargin 0in
\textwidth 6.5in
\textheight 600pt

\begin{document}
\pagestyle{empty}
\title{Yapp 3.0.15 Administration Guide}
\author{Armidale Software}
\maketitle
\newpage
\mbox{}
\newpage
\pagenumbering{roman}
\pagestyle{headings}
\tableofcontents
\newpage

\pagenumbering{arabic}
\chapter{Introduction}
%{
   The Yapp conferencing system can be accessed from Unix and from the
   World Wide Web.  Yapp can be configured to meet the needs of your 
   users, and to create the look and feel you want for your conferencing 
   system.

   The Yapp conferencing system is composed of a set of 
   ``conferences\index{conference}'', each
   devoted to some general topic such as sports or computer games.
   Conferences are referenced by their name, and can only be created by
   a Yapp administrator.  Users who have administrative control over a
   specific conference are called ``fairwitnesses\index{fairwitness}''.
   
   Within each conference, a series of ``items\index{item}'' can exist, 
   each devoted to some specific subject, such as ``Detroit Tigers''.  
   Items are usually referenced by a number.  Conferences can be 
   configured such that items can be created by users.
 
   Each item consists of a message followed by zero or more 
   ``responses\index{response}'' to the original message.  Responses are 
   also referenced by a number, and can be entered by users.
%}

\chapter{Getting Started} \label{c:start}

\section{Installing Yapp from scratch}~\label{s:install}
%{

   To get started running Yapp, you need to have the following:
   \begin{itemize}
   \item a Yapp script distribution\index{generic distribution}.
   \item a Yapp binary distribution\index{binary distribution} for your platform.
           Binaries for several operating systems are available at:
           \begin{verbatim}
           ftp://armidale.ann-arbor.mi.us/pub/yapp/
           \end{verbatim}
           \vspace{-12pt}
   
   \item a license\index{license} provided by Armidale Software.
         To obtain a Yapp license, send email to:

            {\centering yapp@armidale.ann-arbor.mi.us \\}
   \end{itemize}
   
   Once you have acquired the above items, you may proceed as follows:
   \begin{enumerate}
   %{
   \item
      Decide on what login to use to own Yapp files.  Typically, a separate 
      account, is created to own the bbs files.  We will refer to this
      account as {\em cfadm}\index{cfadm}.  Access to this account is needed 
      for a Yapp administrator to create and delete new conferences.  We
      will refer to anyone with access to the {\em cfadm} account as the
      ``Yapp administrator''.

   \item Log in as {\em cfadm}, and uncompress and untar the generic 
      distribution and the binary distribution in the destination directory 
      (e.g. /usr/bbs)  We will herafter refer to this directory as
      {\em bbsdir}\index{bbsdir}.  For example:
      \begin{verbatim}
      $ cd /usr/bbs 
      $ gunzip yapp.3.0.12-generic-dist.tar.gz 
      $ tar -xvf yapp.3.0.12-generic-dist.tar
      $ gunzip yapp3.0.12-solaris-bin.tar.gz
      $ tar -xvf yapp3.0.12-solaris-bin.tar
      \end{verbatim}
      \vspace{-12pt}

   \item While still logged in as {\em cfadm}, run the 
      Install\index{Install} script from the {\em bbsdir} directory.
      This will prompt you for the remaining items and may give you
      additional information about what needs to be done.  The default
      value will be shown in square brackets.  If you wish to use the
      defaults shown, you may simply hit enter at the prompts.
      \begin{verbatim}
      $ ./Install
      \end{verbatim}
      \vspace{-12pt}

      \begin{description}
      \item[{Login to be used as bbs owner [cfadm\index{cfadm}]:}] 
         \mbox{}\newline
         This should be the {\em cfadm} login, and must be the same as the 
         login you are running the script as.

      \item[{Unix login to be used as alternate user administrator[root]:}]
         \mbox{}\newline
          This allows for the administrator of the WWW acounts to be 
          someone other than root.  This is the person who can change
          passwords, enable/disable accounts, etc.

      \item[{WWW login to be used as sysop [sysop]:}] 
         \mbox{}\newline
         This is the login which 
         will be able to access Yapp administrative functions from the web.
         (We will refer hereafter to this login as {\em sysop}\index{sysop}.)
         It may be the same as {\em cfadm}.

      \item[{Unix login used by httpd to invoke programs as [nobody]:}] 
         \mbox{}\newline
         This is the login (typically ``nobody\index{nobody}'') which httpd 
         uses to execute CGI scripts.  (We will refer hereafter to this login 
         as {\em nobody}.) Its value can be found for the NCSA and Apache web 
         servers by looking up the value of User in the web server's 
         httpd.conf file.

      \item[{Enter bbsdir\index{bbsdir} [/usr/bbs]:}] 
         \mbox{}\newline
         This is the main Yapp subdirectory, and should be the same as the 
         current directory when invoking the Install script.

      \item[{Enter wwwdir\index{wwwdir} [/usr/bbs/www]:}] 
         \mbox{}\newline
         This is the subdirectory under which Yapp will put all web-related 
         files and CGI script directories.

      \item[{Enter licensedir\index{licensedir} [/usr/bbs/license]:}] 
         \mbox{}\newline
         This is the subdirectory which Yapp will use for your license 
         information.

      \item[{Enter confdir\index{confdir} [/usr/bbs/confs]:}] 
         \mbox{}\newline
         This is the subdirectory under which Yapp will look for individual 
         conference directories.

      \item[{Enter userhome\index{userhome} [/usr/bbs/www/home]:}] 
         \mbox{}\newline
         This is the subdirectory under which Yapp will keep home directories 
         for web-only users.

      \item[{Enter userfile\index{userfile} [$\sim$/passwd]:}] 
         \mbox{}\newline
         This is where Yapp will keep miscellaneous user information for 
         web-only users.  If the filename begins with a `$\sim$', Yapp will 
         use a separate file for each web user, stored in their home 
         directory.  Otherwise, Yapp will keep a single shared file in the 
         indicated location, with one line per web user.

      \item[{Enter passfile\index{passfile} [/usr/bbs/etc/.htpasswd]:}] 
         \mbox{}\newline
         This is the password file used for web logins, where encrypted 
         passwords are stored.

      \item[{Enter sendmail\index{sendmail} [/usr/lib/sendmail]:}]  
         \mbox{}\newline
         This is the location of the {\tt sendmail} program on your system.  
         Yapp attempts to find the correct location itself and displays it 
         as the default.

      \item[{Enter maildir\index{maildir} [/var/mail]:}] 
         \mbox{}\newline
         This should be set to the system mail directory.  Yapp will look 
         for the user's mailbox there and inform the user when new mail 
         arrives.

      \item[{Allow hosts to freeze items linked to other conferences? [true]:}]
         \index{freezelinked}
         \mbox{}\newline
         Yapp allows an item to simultaneously exist in multiple conferences,
         each controlled by their own fairwitnesses (hosts).  By default,
         a host in any of those conferences will be able to do administrative
         functions on the item.  You can disable this feature by answering 
         {\bf false}.

      \item[{Allow users to censor/scribble responses in frozen items? [true]:}]
         \index{censorfrozen}
         \mbox{}\newline
         When this feature is enabled, users can censor, scribble, and edit
         their own responses in frozen items.  You can disable this feature 
         by answering {\bf false}.

      \item[{Use compressed DBM file for user information? [false]:}]
         \index{userdbm}
         \mbox{}\newline
         This refers to the ``userfile'' mentioned above, and is only valid
         when separate user files are kept per user.  Answering {\bf false}
         means that information is kept as plain text.  Answering {\bf true}
         tells Yapp to keep information in DBM format.  Note that 
         {\tt webuser}\index{webuser} does not currently support DBM 
         format files.

      \item[{Participation file directory? [/usr/bbs/part]:}]
         \index{partdir}
         \mbox{}\newline
         Yapp can allow a user to log in in via either Unix or the WWW and
         maintain a single set of data indicating what the user has 
         previously read.  However, if such combining is used, compatibility 
         with PicoSpan is lost.  If you don't run PicoSpan, just hit return.  
         If you do run PicoSpan, you should enter the word {\bf work}.

         If you are running the Install script after upgrading~\index{upgrade}, 
         and you are changing from using the {\bf work} directory to a 
         single set of data accessable from either Unix or the WWW then 
         you will see the following message:

   \begin{verbatim}
       It looks like you are changing from using user-owned participation
       files, to using cfadm-owned participation files.  If you wish to 
       retain existing information, any participation files under users' home 
       directories must be placed under the new directory.  Do you want
       to have this automatically done now? [yes]
   \end{verbatim}

         If you type {\em no} then you will need to manually move any files that
         you want to have migrated to the new directory structure.

      \item[{Padding\index{padding} size? [78]:}]
         \mbox{}\newline
         Yapp allows authors to edit their item subjects and response text
         as long as the new text fits into the old space.  The padding option
         tells Yapp to leave extra space when entering subjects and responses
         so that there is extra leeway when editing.  Be aware that this will
         cause older Yapp binaries to display ``,E'' at the end of responses, 
         so you should currently use a padding size of 0 if you require 
         backward compatability.

      \item[Changing to cfadm-owned participation files]
         \mbox{}\newline
         If you are changing from using user-owned participation
         files to using {\em cfadm}-owned participation files, and you wish 
         to retain existing information about what users have already seen, 
         then participation files under users' home directories must be 
         placed under the new directory subtree.  If you want the Install
         script to automatically do this for you, hit return at the prompt.
         Otherwise, answer {\bf no}.
      \end{description}

      Once you have answered the above questions, the Install script will 
      attempt to install the options you have requested.  Afterwards, it
      will report any outstanding installation steps which must be performed
      by root.  Those items are described below.

   \item If directed to do so by the Install script, copy or link the 
      file ./yapp.conf\index{yapp.conf} into one of the following places:

      \begin{tabular}{l}
      /etc/yapp.conf \\
      /usr/local/etc/yapp.conf \\
      /usr/bbs/yapp.conf \\
      $\sim$cfadm/yapp.conf\\
      \end{tabular}

   \item If directed to do so by the Install script, copy or link 
         {\em bbsdir}/bin/bbs to /usr/local/bin/bbs, where it should be 
         mode 4711, and owned by {\em cfadm}.

         Additionally (if directed to), you may want to copy or link 
         {\em bbsdir}/webuser\index{webuser} to /usr/local/bin/webuser.  It 
         should be mode 4711 and owned by root.  (See section~\ref{s:webuser} 
         on webuser for more information.)

   \item The man pages for bbs and webuser are in the {\em bbsdir}/man 
         directory.  These web pages can be installed in a standard man 
         directory if you wish.  For example:
         \begin{verbatim}
         $ cp /usr/bbs/man/* /usr/bbs/man/man1
         \end{verbatim}
         \vspace{-12pt}
   
         You may also print the Yapp Manual\index{Yapp manual} by running the 
         {\tt printman}\index{printman} command in the {\em bbsdir}/help 
         directory.  For example:
         \begin{verbatim}
         $ cd \usr\bbs
         $ printman | lpr
         \end{verbatim}
         \vspace{-12pt}
   
   \item You should install the license you obtained from Armidale Software in
      the license directory ({\em bbsdir}/license).  The file should be called
      {\em bbsdir}/license/registered.  It should be owned by {\em cfadm},
      mode 644.

   \item 
   %{
      Tell your web server about Yapp:

      \subsubsection*{Configuring the NCSA or Apache HTTP servers}
      \index{httpd!NCSA} \index{httpd!Apache} 
      %{
         Add the /yapp-bin/ script alias to your server configuration 
         file (srm.conf) where {\em bbsdir} is the appropriate directory for 
         your system): 
         \par

            \qquad \qquad  ScriptAlias /yapp-bin/ {\em bbsdir}/www/cgi-bin/

         \par

         Add the /yapp-icons/ alias to your server configuration file 
         (srm.conf) where {\em bbsdir} is the appropriate directory for 
         your system): 
         \par

            \qquad \qquad  Alias /yapp-icons/ {\em bbsdir}/www/gifs/
         \par

         After configuring your HTTP server, restart the httpd daemon.
      %}

      \subsubsection*{Configuring the CERN HTTP server}
      \index{httpd!CERN} 
      %{
         The Cern server requires a separate Exec directive for each directory 
         containing cgi programs. So, you need to put the following lines in 
         your server configuration file (where {\em bbsdir} is the appropriate 
         directory for your system): \\
{   
      \qquad Exec    /yapp-bin/restricted/*  {\em bbsdir}/www/cgi-bin/restricted/*\\
      \qquad Exec    /yapp-bin/public/*      {\em bbsdir}/www/cgi-bin/public/*\\
}

         Since the cgi-bin directory may not be within the directory tree for 
         the server you may need to follow these steps: 
         \begin{enumerate}
         %{
            \item To allow everyone with a password to access Yapp's restricted 
                  directory, set your configuration as follows. 

                  {\centering         
                  \begin{tabular} {lll}
                  Protection&   YAPP    \{\\
                  AuthType&     Basic\\
                  ServerID&     yapp\\
                  PasswordFile& {\em bbsdir}/etc/.htpasswd\\
                  GetMask&         All\\
                  \}& \  \\
                  \end{tabular}

                  Protect /yapp-bin/restricted/*  YAPP\\
                  }

                  If you would like to define a group of users who may have 
                  access to the restricted directory you must create a 
                  separate groupfile.  For more information, see Cern's 
                  documentation at:

   {\centering http://www.w3.org/pub/WWW/Daemon/User/Config/AccessAuth.html\\ }

            \item In the Yapp {\em bbsdir}/www/cgi-bin/restricted directory, 
                  you must create a file called 
                  {\tt .www\_acl}\index{.www\_acl}. That file should contain 
                  the line: 

                     {\centering * : GET,POST : All\\}

                  This will allow all users with passwords access to the 
                  restricted directory.
         %}
         \end{enumerate}

         After configuring your HTTP server, restart the httpd daemon.
      %}
   %}

   \item The following are links which you may wish to add to your own pages.
      %{
         \begin{description}
         %{
         \item [The main menu of Yapp:]
             /yapp-bin/restricted/main
   
         \item [The newuser login page:]
             /yapp-bin/public/newuser
   
         \item [An index of read-only conferences:]
             /yapp-bin/public/index
         %}
         \end{description}

         Example HTML excerpt:
         \begin{verbatim}
      <ul>
      <li><A HREF="/yapp-bin/public/newuser">Register as a new user</A><P>
      <li><A HREF="/yapp-bin/restricted/main">Log in as an existing user</A><P>
      <li><A HREF="/yapp-bin/public/list">View read-only conferences</A>
      </ul><P>
         \end{verbatim}
      %}

%DT9/9/96
% I removed this because it's really part of installing httpd, not
% installing Yapp.
%   \item If one does not already exist, create an account for the 
%      {\em nobody}\index{nobody} login you specified when running the 
%      Install\index{Install} script.  

   \item Optionally create a mail alias to use with mailing 
         list~\index{mailing list} conferences by adding the following 
         line to /etc/aliases.\index{cflink}\\

         cflink:``\vline \ /usr/local/bin/bbs -i'' \\

         Then to activate the new alias do:
         \begin{verbatim}
         $ newaliases
         \end{verbatim}
         \vspace{-12pt}

         See section~\ref{s:maillist} for more information on configuring 
         mailing list conferences.

   \item If the Yapp user home directory is first created by the Install
         script, it will not be owned by the {\em nobody}\index{nobody} 
         login.  The Install script will tell you to chown this directory 
         if it needs to be done (e.g. ``As root: chown nobody 
         /usr/bbs/www/home'').  For example:
         \begin{verbatim}
         $ chown nobody /usr/bbs/www/home
         \end{verbatim}
         \vspace{-12pt}

   \item (Solaris\index{Solaris} only) When a script's shell is a setuid 
      program, Solaris doesn't do the setuid\index{setuid}.  This is a 
      problem for Yapp scripts, since Yapp must run setuid {\em cfadm}.  
      A workaround is to make all Yapp scripts in 
      {\em bbsdir}/www/cgi-bin/public and 
      {\em bbsdir}/www/cgi-bin/restricted setuid cfadm.  If you are running
      Yapp on Solaris, you need to make all Yapp scripts be owned by 
      {\em cfadm}, mode 4755.
         \begin{verbatim}
         $ cd /usr/bbs/www/cgi-bin/restricted
         $ chown cfadm *
         $ chmod 4755 *
         $ cd /usr/bbs/www/cgi-binn/public
         $ chown cfadm *
         $ chmod 4755 *
         \end{verbatim}



   \item If the {\em sysop} and {\em cfadm} logins are different, you should
         immediately go through the newuser process (at 
         http://localhost/yapp-bin/public/newuser) and create the {\em sysop}
         web account.  This login will be able to access administration
         functions from the WWW.

   \item You may wish to customize the WWW pages of Yapp.  See the System
         Configuration section \ref{s:sysconfig} for more information.

   %}
   \end{enumerate}   
%}

\section{Upgrading from Yapp 2.X}
%{
   \index{upgrade}
   Before doing the installation, you should make a backup of all 
   of your Yapp files as they currently exist.  This includes all
   files located in any subdirectories of your Yapp directory 
   ({\em bbsdir}).
   
   You will also need to obtain a new license file from Armidale Software 
   (send email to yapp@armidale.ann-arbor.mi.us).  Yapp 3.0 will
   not accept a Yapp 2.X license.

   Upgrading from Yapp 2.X can then be done using the same installation 
   procedure described in section~\ref{s:install}.  Your existing conferences 
   should be unaffected by the installation process.  

   If you need help with your upgrade, contact your provider for Yapp.
%}

\section{Upgrading from Yapp 3.0 -- 3.0.15}
%{
   \index{upgrade}
   \subsection{Preparation}
   %{
      If you need help with your upgrade, contact your provider for Yapp.
   
      \begin{enumerate}
      %{
   
         \item Make a backup of all of your Yapp files as they currently exist.
               This includes all files located in any subdirectories of your 
               Yapp directory ({\em bbsdir}).
   
         \item Determine the current version of your binary and scripts.
               One way to accomplish this is to go to the debug~\index{debug} 
               pages provided with Yapp. Check the page:
      
               http://localhost/yapp-bin/public/debug
   
         \item If you are running Yapp 3.0.10 or earlier and are upgrading to 
               Yapp 3.0.11 or later, and you are running on any of the following
               platforms: Solaris,\dots  then you need to obtain a new license
               file (send email to yapp@armidale.ann-arbor.mi.us).
   
         \item Obtain the generic distributions assocated with your current 
               version of Yapp, and the version to which you wish to upgrade. 
               Also obtain the binary distribution for the version you wish to 
               upgrade.  If there is no new generic distribution associated with 
               the binary upgrade, ignore the instructions for the generic 
               distribution upgrade.  The distribution files should be 
               available from:
   
               {\centering ftp://armidale.ann-arbor.mi.us/pub/yapp}
   
               FTP the files to your Yapp bbs directory ({\em bbsdir}).
      %}
      \end{enumerate}
   %}
   
   \subsection{Upgrading the binaries}
   %{
      \begin{enumerate}
      %{
      \item {\em Note: This next step may temporarily cause your BBS to stop
         functioning properly.  This is because you will be in the middle 
         of an upgrade.}
   
         Log in as the Yapp administrator ({\em cfadm}), and extract the
         new binary distribution in the {\it bbsdir} directory.
   
         For example if you are upgrading to Yapp-3.0.12 on a BSDI platform:
         \begin{verbatim}
         $ cd /usr/bbs
         $ tar xvf yapp3.0.12-bsdi-bin.tar
         \end{verbatim}
         \vspace{-12pt}
   
      \item From the {\em bbsdir} directory run the Install script.  
         For example:
         \begin{verbatim}
         $ ./Install
         \end{verbatim}
         \vspace{-12pt}
   
         For more information on the Install script see section~\ref{s:install} 
         on installing Yapp, and section~\ref{s:yapp.conf} on yapp.conf.  
         The defaults listed will be your former settings, if you had former
         default settings.  You are running the Install script again because 
         there may now be additional defaults that you need to set.
   
      \item If you did not see a line similar to:
   
            ``Successfully installed /etc/yapp.conf''
   
         then you need to copy the file {\em bbsdir}/yapp.conf to the directory 
         where your yapp.conf file is kept.  This may require you to be 
         logged in as root.  See section~\ref{s:yapp.conf} on yapp.conf
         to see the directories in which this file may be placed.
   
      \item Now copy or link the new Yapp binary to the /usr/local/bin 
         directory.  You may need to be logged in as root to do this.
   
         For example:
         \begin{verbatim}
         $ ln -s /usr/bbs/bin/bbs /usr/local/bin/bbs
         \end{verbatim}
         \vspace{-12pt}
   
         Make sure that /usr/local/bin/bbs is owned by the conference 
         adminstator ({\em cfadm}) and that it is mode 4711.
   
         For example, to change the owner and the mode (you will need to be
         logged in as root):
         \begin{verbatim}
         $ chown cfadm /usr/local/bin/bbs
         $ chmod 4711 /usr/local/bin/bbs
         \end{verbatim}
         \vspace{-12pt}
   
         You may also wish to copy {\em bbsdir}/bin/webuser into /usr/local/bin.  
         See the section on webuser, and the man page for webuser in
         Appendix~\ref{a:manpages} for more information on webuser.
   
      \item The man pages for bbs and webuser are in the {\em bbsdir}/man 
         directory.  These web pages can be installed in a standard man 
         directory if you wish.  For example:
         \begin{verbatim}
         $ cp /usr/bbs/man/* /usr/bbs/man/man1
         \end{verbatim}
         \vspace{-12pt}
   
         You may also print the Yapp Manual\index{Yapp manual} by running the 
         {\tt printman}\index{printman} command in the {\em bbsdir}/help 
         directory.  For example:
         \begin{verbatim}
         $ printman | lpr
         \end{verbatim}
         \vspace{-12pt}
      %}
      \end{enumerate}
   %}
   
   \subsection{Upgrading the scripts}
   %{
      \begin{enumerate}
      %{
   
      \item While logged in as the Yapp administrator ({\em cfadm}), 
         run the Upgrade script located in your Yapp directory ({\em bbsdir}).
         

         \begin{verbatim}
         $ ./Upgrade
         \end{verbatim}

         If you do not have an Upgrade script, you can ftp one from :
         \begin{verbatim}
         ftp://armidale.ann-arbor.mi.us/pub/yapp/Upgrade
         \end{verbatim}
         \vspace{-12pt}

      \item You will receive a list of files which could not easily be 
         upgraded when the script is finished.  You will need to manually patch
         the files which are listed.
   
         You should view the contents of each rejected patch (filename.rej) 
         and the file you were trying to patch at the same time.  
         You can do this by using multiple screens, or by printing the patch 
         file.
   
         While viewing a patch file (or filename.rej):

         \begin{tabular}{ll}
         ``-'' & Means that the indicated line is deleted\\
         ``+'' & Means that the indicated line is added\\
         ``!'' & Means that the indicated line was changed\\
         \end{tabular}
   
         For more infomation on reading the patch file, see the man page for
         diff(1).   The patch file is in standard {\tt diff -c} format.
   
         You will see for each change, the original entry on the top, 
         and your changes underneith the --\#,\#-- line.  You will need to 
         determine what the conflicts are between your changes, 
         and the existing new file.  Your original file is saved in 
         filename.orig.
   
      \item  Manually change the distribution file to incorporate any changes 
         you have made.
   
         When you are done changing the distribution file, remove the 
         files filename.rej and filename.orig. 
          
      \item Check to make sure you have made all of your changes to the new
         distribution.  The BBS should be working again, and you should
         be able to test all WWW pages, and to use the Unix version of bbs.
   
         If you notice a bug in the new distribution, send email to:
            \begin{verbatim} 
            yapp@armidale.ann-arbor.mi.us.
            \end{verbatim}
      %}
      \end{enumerate}
   %}
%}

\chapter{User Administration} \label{c:user}
%{
   \section{Creating WWW accounts for existing Unix users} \label{s:webuser}
   %{
       In the binary distribution\index{distribution}, there is a Unix command 
       called {\tt webuser}\index{webuser}.  Webuser without any options will 
       create a web account for an existing Unix account.  To allow all 
       Unix users automatic access to the web using their Unix login and 
       password, {\tt webuser} should be run when the Unix account is created.
   
    
       For more information on webuser, see the man page for webuser in 
       Appendix~\ref{a:manpages}.
   %}

   \section{Deleting WWW accounts}
   %{
      To delete WWW accounts you should run the Unix command 
      {\tt webuser -r {\em login} }\index{webuser}  This will remove the 
      directory and all account information for {\em login}.  You must be
      either be root or the User Administrator\index{User Administrator} 
      which we will refer to as {\em usradm}\index{usradm} if one is 
      defined in the yapp.conf file.
   %}

   \section{Resetting the password on a WWW account}
   %{
      The password on a WWW account can be reset by running the Unix command
      {\tt webuser -p {\em login} }\index{webuser}.  This will change only a 
      web password for {\em login}. You must either be root or {\em usradm} to 
      change the password of an arbitrary login.

      If a Unix user executes {\tt webuser -p}\index{webuser} it will update 
      the web password assocated with the user's login, as long as the 
      web account already exists.
      
      You may wish to create a wrapper script around the Unix command 
      {\tt passwd} so that whenever a user updates their Unix password,
      {\tt webuser}\index{webuser} is called to update their web password 
      as well.

   %}
   \section{Enabling and disabling a WWW account}
   %{
      As root or {\em usradm}\index{usradm} you have the ability to 
      enable or disable WWW accounts.  When a WWW account is created the 
      status is set to enabled.

      To disable an existing account run the Unix command
      {\tt webuser -d {\em login}}\index{webuser}.  This will not 
      perminently remove the user's files, but the user will no longer 
      be allowed access via the WWW.

      To enable an account which has been disabled run the Unix command
      {\tt webuser -e {\em login}}\index{webuser}.  This will allow the 
      user access via the WWW which was previously restricted.
      
   %}

   \section{Getting a status report on a WWW account}
   %{
      A status report on a WWW account will show you the login, 
      enabled/disabled status, full name, email address, and the time they
      last accessed their account.  To get a status report on an individual
      account run the Unix command 
      {\tt webuser -s {\em login}}\index{webuser}.  
      
   %}

   \section{Listing all WWW accounts}
   %{
      A listing of all WWW accounts will show you the login, date of last
      access, email address, and full name of each WWW user.  This list can
      be generated by running the Unix command {\tt webuser -l}\index{webuser}.
      Specifing {\em login} on the command line does not restrict the 
      report to a single user.
   %}
%}
 
\chapter{Conference Administration} \label{c:conf}
%{
   \section{Creating conferences} \label{s:create}
   %{
      There are three ways to create conferences.  
   
\renewcommand {\theenumi}{\Alph{enumi}}
\renewcommand {\labelenumi}{\theenumi.}

      \begin{enumerate}
      %{
         \item From a Yapp prompt, run the command {\tt cfcreate}
            if you are logged in as {\em cfadm}
   
         \item Log into Unix as the conference adminstrator ({\em cfadm}).  Run
            {\tt cfcreate} from the Unix prompt
   
         \item  From the WWW, you can access the {\em Create a conference} 
            link on the {\em Index of Conferences} page while logged in as 
            {\em sysop}.
      %}
      \end{enumerate}

\renewcommand {\theenumi}{\arabic{enumi}}
\renewcommand {\labelenumi}{\theenumi.}

      In any method you will be prompted for the following:
      \begin{description}
      \item[A short conference name]
         \mbox{}\newline
         A short name is the conference name, including the underscore to 
         indicate the minimum name which must be typed to access the 
         conference.  
         For example, soft\_ware could be the short name of the ``software'' 
         conference.  When joining the conferance ``soft'', ``softw'', 
         ``softwa'', ``softwar'', and ``software'' would all be equivilent.

      \item[A one line description]
         \mbox{}\newline
         The one line description is what gets displayed as the description 
         of the conference when listing the index of conferences.

      \item[The conference directory]
         \mbox{}\newline
         The conference subdirectory to be created by Yapp.  This directory 
         will house the conference related files.

      \item[Logins or UIDs of fairwitnesses]
         \mbox{}\newline
         The ``fairwitnesses'' are the ones who administrate the needs of a 
         conference on a day to day basis. You need not list {\em cfadm} as a 
         fairwitness, since it is automatically a fairwitness in every 
         conference.

      \item[The conference security type]
         \mbox{}\newline
         The conference type is kept in the config file for a conference, and 
         can only be changed by {\em cfadm}.  It can be set either 
         numerically, or with a combination of keywords.\\

         {\em NOTE: only the MAILLIST and REGISTERED flags are 
              significant when a conference acl\index{acl} file exists.  The 
              others are obsoleted by the acl system. See section~\ref{s:acl} 
              or the Yapp manual on ``file acl'' for more information.
         }

\begin{tabular}{llp{7cm}}
   Type      & Code  & Description\\ \hline
   PUBLIC    & 0     & Public Conference \\
   PRESELECT &   4   & Access restricted to user list (in `ulist' file)\\
   PASSWORD  &   5   & Access requires password (in `secret' file)\\
   PARANOID  &   6   & Use both a user list and a password together\\
   PROTECTED &   8   & Public except item files are made mode 600\\
   READONLY  &  20   & Anyone not in ulist may observe\\
   READPASS  &  21   & Anyone who fails the password check may observe\\
   READPARA  &  22   & Anyone who fails either check may observe\\
   MAILLIST  &  \#+64 & Mailing list conference (usually code 72)\\
   REGISTERED&  \#+128& Registered mailing list conference (usually code 200)\\
   NOENTER   &  \#+256& Only fairwitnesses may enter new items\\ \hline
\end{tabular}
\vspace{12pt}

         Keywords are:

         \begin{tabular}{lr}
            public     &  0\\
            ulist      &  4\\
            password   &  5\\
            protected  &  8\\
            readonly   & 16\\
            maillist   & 64\\
            registered &128\\
            noenter    &256\\
         \end{tabular}

         A ``registered'' maillist~\index{maillist} conference is one in 
         which only mail from registered users gets posted to the conference.  
         If the sender's email address does not match the email address 
         of some user known to Yapp, the email is silently dropped.

      \item[Let a fairwitness change the access control list?]
         By default only the Conference Administrator can change the 
         access control list.  If you wish fairwitnesses of the conference
         to be able to change the security answer yes.  You must answer
         either yes or no to this question.

      \item[Email address(es) (if the conference is linked to a mailing list)]
         \mbox{}\newline
         The email addresses are the addresses which the conference will send
         mail to, and recieve mail from.  See section~\ref{s:maillist} 
         for more information on configuring a mailing list conference.  The
         addresses must be comma separated.
      \end{description}

%      \subsubsection{}
         When a conference is created the following things happen:
         \begin{enumerate}
         %{
         \item an entry is added to the {\em bbsdir}/conflist file
         \item an entry is added to the {\em bbsdir}/desclist file
         \item the subdirectory you specified is created
         \item the conference config file is created, and placed in the 
            subdirectory
         \item the login file (displayed when a user logs into the conference 
            from Unix) is created and placed in the subdirectory
         \item the logout file (displayed when a user logs out of the conference 
            from Unix) is created and placed in the subdirectory
         %}
         \end{enumerate}
   %}

   \section{Deleting conferences} \label{s:delete}
   %{
      There are three ways to delete\index{delete conference} conferences.  

\renewcommand {\theenumi}{\Alph{enumi}}
\renewcommand {\labelenumi}{\theenumi.}

      \begin{enumerate}
      %{
         \item From a Yapp prompt run the command ``cfdelete'' \index{cfdelete}
            if you are logged in as {\em cfadm}.

            If no conference name is specified after the command, Yapp will
            prompt for a conference name.  If the conference you specify is not 
            empty, it will not be deleted and Yapp will simply issue an error
            message.  To delete a conference with items in it, you must first 
            {\tt kill} all the items.

         \item Log into Unix as the conference administrator ({\em cfadm}) and 
            run cfdelete from a Unix prompt.

            Yapp will then prompt for a conference name.  If the conference
            you specify is not empty, it will not be deleted and Yapp will 
            simply issue an error message.  To delete a conference with items 
            in it, you must first {\tt kill} all the items.
   
         \item From the WWW, you can access the {\em Delete Conference} link on 
            the {\em Index of Conferences} next to the Conference name if you
            are logged in as {\em sysop}\index{sysop}.
   
            The {\em Delete Conference} link will not appear on the Index of 
            Conferences page while the conference still contains items.  The 
            Kill button is available on the item read page and is used to remove 
            the items in a conference before it can be deleted.
      %}
      \end{enumerate}
   
\renewcommand {\theenumi}{\arabic{enumi}}
\renewcommand {\labelenumi}{\theenumi.}

      When a conference is deleted, the following happens:
      \begin{enumerate}
      \item Its entry is removed from the conflist\index{conflist}
      \item Its entry is removed from the desclist\index{desclist}
      \item The conference directory and contents are removed 
      \item If a common participation file directory 
            ({\em partdir}\index{partdir}) is used, all members'
            participation files will be removed, and the conference will be
            removed from their conference hotlist if necessary.
      \end{enumerate}
   %}
   
   \section{Modifying the options on an existing conference} \label{s:modify}
   %{
      \subsection{The acl file} \label{s:acl}
       %{
         The access control list (acl\index{acl}) contains the security
         requirements for a conference, and are modifiable by a
         conference administrator.  The file can be change from Unix with the
         {\tt change acl} command, or from the WWW from the fairwitness
         page.  If the acl file exists, it overrides the security options 
         in the conference type in the config file.
         
         The acl\index{acl} file should consist of ``r'', ``w'', 
         ``c'', and ``a'' lines, which may occur in any order.  The ``r'' line 
         specifies who may read items in the conference (and thus, who 
         may join the conference).  The ``w'' line specifies who may write 
         responses to existing items.  The ``c'' line specifies who may 
         create new items. The ``a'' line specifies who can edit the 
         acl file.
        
         Each line consists of the type (r/w/c/a), followed by one or more
         fields (in any order) from the following list, each optionally 
         prefixed with a `+' or a `-':
         \par
         \begin{tabular}{lp{4.5in}} 
            Field Name                   & Description \\ \hline
            all                          &  Anyone \\
            registered                   &  Users with accounts \\
            fwlist                       &  Fairwitnesses \\
            originlist\index{originlist} &  Anyone passing an origin check \\
            password\index{password}     &  Anyone who knows the conference 
                                            password \\
            f:ulist\index{ulist}         &  Anyone listed in the ulist file \\
            sysop\index{sysop}           &  Conference Administrators \\
            f:observers\index{observers} &  Anyone listed in the 
                                            observers file \\
            f:{\tt filename}             &  Anyone listed in the 
                                            {\tt filename} file \\
         \end{tabular}
      \vspace{12pt}


         Conferences which require a password as part of the security are
         not currently supported from the WWW.
       
         If a +, or nothing, precedes the field, a user must satisfy the
         indicated condition for permission to be granted.  If a - precedes 
         the field, a user must NOT satisfy the indicated condition.
         In order for permission to be granted, a user must satistify ALL
         conditions in the appropriate line.  
      
      Example:
      {\tt
         \begin{verbatim}
         r +all +registered
         w +all +registered -f:observers
         c +all +registered +fwlist
         a +sysop
         \end{verbatim}
      }
      
         In the example file above, any registered user may join the conference
         and read items.  All registered users except those listed in the 
         ``observers'' file are able to respond.  Only fairwitnesses are able
         to enter new items, and only the Conference administrator may change
         the acl file.  If no ``a'' line is included, it defaults to 
         ``a + sysop''.
     
         For more information see section~\ref{s:ulist} on the ulist file, 
         section~\ref{s:observers} for the observers file, 
         section~\ref{s:origin} for information on the origin list,
         and section~\ref{s:secret} for more information on 
         password protected conferences.
      %}

      \subsection{The ulist file} \label{s:ulist}
      %{
         The ulist\index{ulist} is a file containing a list of people 
         who are authorized to be in a conference.  For private conferences 
         which have a fixed user list, this list should be created and 
         maintained by a fairwitness.  For public conferences, this 
         file is automatically maintained by Yapp.  It contains a list of 
         logins and/or UIDs of participants separated by whitespace 
         (spaces, tabs, or newlines).  Any lines beginning with a 
         ``\#'' character are interpreted as comments, and ignored. 
 
         A filename may also appear in the ulist, signifying that the
         contents of filename should also be considered as part of the
         ulist.  This allows for one master list to be maintained for
         several conferences, or for a conference ulist to be the union
         of the individuals listed in several different files.

         For more information see {\tt file ulist} and {\tt change ulist} in 
         the Yapp Manual.

      %}

      \subsection{The observers file} \label{s:observers}
      %{

         An observers file \index{observers file} can be maintained for 
         each conference, and can only be edited by a fairwitness.
         To edit this file from Unix, type {\tt change observers}
         at the {\tt ok} prompt.  This file can also be edited from the
         fairwitness page on the WWW.  The observers file contains a list 
         of users, specified by either login or UID, and separated by 
         whitespace.  

         A filename may also appear in the observers file, signifying that the
         contents of filename should also be considered as part of the
         observers list. 

         This user list can be referenced in the acl file to allow (for example)
         preventing the indicated users from responding in the conference.
         See section~\ref{s:acl} for more information on the acl file.

      %}

      \subsection{The originlist file} \label{s:origin}
      %{

        The originlist\index{originlist} file is used for conferences 
        in which partipation is restricted to users on machines in a 
        particular domain or IP address prefix.  It contains a list 
        of hostnames, domain names, IP addresses and/or subnet prefixes 
        separated by whitespace (spaces, tabs, or newlines).  Any lines 
        beginning with a "\#" character are interpreted as comments, and 
        ignored.  For a Unix or web user to be allowed to join the 
        conference, their host must match one of the hostnames.  This file 
        must reside in the conference directory and can be edited from 
        the WWW from the fairwitness page.  It can also be change
        from Unix with the command {\tt change originlist}.

        \begin{tabular}{lp{4.5in}} 
           Example & Description \\ \hline
           ann-arbor.mi.us  &  allows any hostname equal to or ending with
                               "ann-arbor.mi.us" \\
           141.211.39.193   &  allows the indicated IP address\\
           141.211.39       &  allows any IP address starting with 
                               "141.211.39" \\
           141.211.38/23    &  allows any IP address in the indicated subnet
                               range (i.e. any address whose high 23 bits match
                               141.211.38) \\
        \end{tabular}

      %}

      \subsection{The secret file} \label{s:secret}
      %{
          A file named {\tt secret} can be maintained in each conference,
          and should consist of a single line containing the
          exact password\index{password} needed to access the conference.  
          This file is only used if the conference security type warrants it, 
          or if it is referanced by the acl file.
 
          This file can only be changed by a fairwitness of the conference
          from Unix.  Password protected conferences are not supported 
          through the WWW.  To change the contents of this file type 
          {\tt change secret} at the {\tt ok} prompt.  

      %}
   
   %}



   \section{Configuring mailing list conferences} \label{s:maillist}
   %{
     \index{maillist}
      A mailling conference is usually configured to both send
      and recieve mail.  The conference is configured to send mail
      if the conference type includes the `maillist' flag, and 
      one or more email addresses are present on line 6 of the conference 
      config file.  Both of these fields are defined when a conference
      is created.  This information can be changed by {\em cfadm} 
      with the {\tt change config} command, or by the {\em sysop} 
      from the conference fairwitness page.

      For more information about these fields see section \ref{s:create} on 
      creating conferences.
   
      If you wanted to create a conference linked to the Yapp mailing list,
      the following would be on line 6 of the Yapp conference config file:\\
   
        {\centering yapp@umich.edu} \\
  
      In order for Yapp to post the information recieved from a mailing list
      to the correct conference the information in {\em bbsdir}/maillist
      needs to be updated. The file maillist\index{maillist} specifies
      which incoming mail addresses go to which conference.  (This is necessary
      because multiple incoming addresses may actually be the same list.) 
      The maillist file is updated automatically when you create a conference,
      or when you change the conference configuration file.
   
      The first line of the {\em bbsdir}/maillist\index{maillist} file should 
      be the string ``!$<$hl01$>$''.
      The second line should contain the directory of the default conference
      which, if it matches a conference name, will collect mail which doesn't
      match anything else.  If it isn't a conference name, excess mail will
      be lost. The remaining lines should consist of an email address, 
      a colon, and a conference name.  A conference may have multiple entries 
      in order to send multiple addresses to the same conference.
   
      Example:\\
      \qquad   !$<$hl01$>$\\
      \qquad  lost\\
      \qquad   yapp@umich.edu:yapp\\
      \qquad   cseg@zip.eecs.umich.edu:cseg\\
      \qquad   cseg@dip.eecs.umich.edu:cseg\\
      \qquad   cseg@quip.eecs.umich.edu:cseg\\
      \qquad   amber@hagar.ph.utexas.edu:amber\\
      \qquad   oberon@amber.uchicago.edu:amber\\
      \qquad   mlist-amber@nntp-server.caltech.edu:amber\\
  

      Although a conference may be configured to recieve mail, if 
      the appropriate mail alias is not set up, or if the mail alias
      is not on the appropriate mailing list Yapp won't
      post any information to the conference.  Information for 
      configuring the mail alias can be found in section \ref{s:install}.
      

      To be placed on the Yapp conference email list, send email to 
         \begin{verbatim} 
         yapp@armidale.ann-arbor.mi.us
         \end{verbatim}
   
      Ask to be placed on the Yapp conference list, and include the email 
      address of the alias you created to receive mail.  
      (``yapp@umich.edu'' is a mailing list maintained at an X.500 
      directory server.  Fingering it will give the current list of members.)  
   
      Once the email address is added to the mail list of interest, your 
      bi-directional link will be complete.
   
      For more information on correcting problems with a maillist conference, 
      see section~\ref{s:tmaillist}.
   %}
%}

\chapter{Configuring the Look and Feel of the BBS} \label {c:configur}
%{
   
  \section{Changing Yapp configuration parameters} \label{s:yapp.conf} 
   %{
      Yapp configuration parameters\index{configuration parameters} are 
      contained in a yapp.conf\index{yapp.conf} file, which is maintained 
      by the Install\index{Install} script (see section~\ref{s:install} on 
      installing Yapp).  The yapp.conf file can only be changed by a Yapp 
      administrator ({\em cfadm}). Yapp searches for this file in the 
      following places (in order):

      \begin{tabular}{l}
       /etc/yapp.conf \\
       /usr/local/etc/yapp.conf \\
       {\em bbsdir}/yapp.conf \\
       $\sim$cfadm/yapp.conf \\
       ./yapp.conf \\
      \end{tabular}
   
      This file consists of zero or more lines of the form:

         {\centering {\em parameter}:{\em value}\\}

      where {\em parameter} is one of those listed below.
      \vspace{12pt}

      \tablefirsthead{Parameter & Description & Default \\ \hline}
      \tablehead{\multicolumn{3}{l}{\small\slshape continued from previous page}\\\hline Parameter & Description &  Default \\ \hline}
      \tabletail{\hline\multicolumn{3}{r}{\small\slshape continued on next page}\\}
      \tablelasttail{\hline}
      \begin{supertabular}{lp{7cm}l}
      %{
          bbsdir\index{bbsdir}             & Base Yapp directory ({\em bbsdir})
             & /usr/bbs\\
          byteswap\index{byteswap}         & Byteswap constant (do not change)&  1 \\
          censorfrozen\index{censor}       &  
             Whether users can censor/scribble responses in frozen items & 
             true \\
          cfadm\index{cfadm}               &  Unix login which ``owns'' Yapp files&  cfadm \\
          confdir\index{confdir}           &  
             Parent directory under which new conferences should be created &   
             {\em bbsdir}/confs\\
          freezelinked\index{freezelinked} &
             Whether fairwitnesses can freeze items  which are linked to 
             other conferences             & true \\
          licensedir\index{licensedir}     & Directory holding license 
             information                   & /usr/bbs/license\\
          maildir\index{maildir}           & Unix mailbox directory 
                                           &  /usr/spool/mail\\
          nobody\index{nobody}             &  Unix login which httpd uses to run 
             CGI programs & nobody\\
          padding\index{padding}           & Subject/response padding size  
                                           & 0\\
          partdir\index{partdir}           & Participation file directory   
                                           & WORK\\
          passfile\index{passfile}         & Full pathname of the .htpasswd 
             file in which httpd looks up web passwords 
                                           & /usr/bbs/etc/.htpasswd\\
          safe\index{safe}                 & Whether to auto set nosource for 
             {\em cfadm} and set observe nosource for root & true\\
          sendmail\index{sendmail}         & Full pathname of Unix sendmail 
             program                       & /usr/lib/sendmail \\
          sysop\index{sysop}               & Web login of Yapp administrator 
                                           & (same as Yapp owner)\\
          userdbm\index{userdbm}           & Use compressed DBM file for 
             userfile?                     & false \\
          userfile\index{userfile}         & Full pathname of file holding 
             extra web user account information 
                                           & /usr/bbs/etc/passwd\\
          userhome\index{userhome}         & Base path of www user home 
             directories                   &  /usr/bbs/home\\
          usradm\index{usradm}             & Unix login to be used as 
             alternate user administrator  & root \\
          wwwdir\index{wwwdir}             & Base path of Yapp's www files 
                                           & {\em bbsdir}/www \\
     %}
     \end{supertabular}
   %}

   \section{Using separators and macros} \label{s:sep}
   %{
      {\em Separator strings}\index{separators} are a way of customizing 
      output.  Separator strings can be saved in variables, many of which 
      have default values.  There are six types of separators:

      \vspace{12pt}
      \begin{tabular}{ll}
      Type & Description \\ \hline
      miscsep& value applies in all cases\\
      confsep& value applies to conference information (based on context)\\
      itemsep& value applies to item information (based on context) \\
      misccond& condition works in all cases\\
      confcond& condition works for conference information only\\
      itemcond& condition works for item information only\\ \hline
      \end{tabular}
      \vspace{12pt}

      In general, each separator code is of the form: \%\#X
      where \# is an optional number , `z' or `$\wedge$'
      and X is some character.  The \# represents the
      width of the field when displayed, or indicates the format 
      of the date in date separators, or a file number when a condition 
      references a file number by definition.  See the entries on 
      {\tt datesep} and {\tt date} in the Yapp Manual for 
      more information about the date formats.

      Some exceptions are:

      \begin{tabular}{lp{10cm}}
      \%$<$var$>$  &   means expand entire variable here (the $<>$ are literals) and
                 capitalize it if $\wedge$ was specified \\

      \%\{var\}  &   means expand entire variable here (the \{\} are literals) and
                 capitalize it if $\wedge$ was specified \\

      \$\{var\}  &   means expand variable here without re-expanding recursively
                 and capitalize it if $\wedge$ was specified \\

      \%(X ... \%)& means expand ... only if X condition holds (note that
                 \%E in ... functions as an ELSE clause) \\

      \%((icon op icon) ... \%) & means expand ... only if the given condition
                 holds.  See syntax for the {\tt if} command for more information. \\

      \%`command` &means replace with output of the given command \\
      \end{tabular}



      \subsection{Conference separators }
      %{

         Conference separators are only applicable in conference separator
   variables.  For any codes which use a number, the number defaults to 0
   unless otherwise stated.  A code is specified with: \%Z\#X 
      where Z is an optional `Z' or `z' meaning print `No' or `no' in place of 0
        \# is an optional number (indicating field width or a file number),
        X is some character below.
   
         \tablefirsthead{Separator & Description \\ \hline}
         \tablehead{\multicolumn{2}{l}{\small\slshape continued from previous page}\\\hline Separator & Description \\ \hline}
         \tabletail{\hline\multicolumn{2}{r}{\small\slshape continued on next page}\\}
         \tablelasttail{\hline}
         \begin{supertabular}{ll}
         %{

            b     &number of brandnew\index{brandnew:separator} items \\
            C     &string text for index command \\
            d     &conference directory \\
            f     &number of first item \\
            g     &dump entire file \# \\
            i     &number of items \\
            k     &number of things processed so far in the current report \\
            l     &number of last item \\
            L     &name of current conference \\
            m     &last modification date of conference (timestamp on 
                   summary file) \\
            n     &number of new (brandnew or newresponse) items \\
            o     &date of last session in conference (timestamp on 
                   participation file) \\
            p     &participation file name \\
            q     &last component of conference directory \\
            Q     &if not in a conference, print message and abort separator \\
            r     &number of newresponse\index{newresponse:separator} items \\
            s     &name of conference \\
            t     &conference security type (number) \\
            u     &your full name in this conference \\
            v     &your login \\
            w     &your Yapp directory (usually your home directory or 
                   .cfdir directory) \\
            y     &number of unseen items                  \\
         %} 
         \end{supertabular} 

         File numbers for separators which access file numbers for a 
         conference are:

         \begin{tabular}{ll}
         %{
            File Number & Description \\ \hline

            0           & login\index{files!login} \\
            1           & logout\index{files!logout} \\
            2           & index\index{files!index} \\
            3           & bulletin\index{files!index} \\
            4           & welcome\index{files!welcome} \\
            5           & htmlheader\index{files!htmlheader} \\
         %}
         \end{tabular} 
      %}

      \subsection{Conference conditionals}
      %{

         Conference conditionals apply only to conference separator
         variables.  For any codes which use a number, the number defaults to 0
         unless otherwise stated.  Codes inside conditionals are used only if
         the condition is true.  A code is specified with: \%(!\#X 
         where ! is an optional `!' or `\~' meaning to negate the condition
         \# is an optional number, and  X is some character below.

         \tablefirsthead{Separator & Description \\ \hline}
         \tablehead{\multicolumn{2}{l}{\small\slshape continued from previous page}\\\hline Separator & Description \\ \hline}
         \tabletail{\hline\multicolumn{2}{r}{\small\slshape continued on next page}\\}
         \tablelasttail{\hline}
         \begin{supertabular}{ll}
         %{
            b     & if there are brandnew\index{brandnew:condition} 
                    items in the conference \\
            B     & if processing the conference you are in \\
            C     & if you are currently in some conference \\
            f     & if summary file exists (for the conference you are in ?) \\
            F     & if \# file exists \\
            i     & if any items exist in the conference  \\
            j     & if you just joined this conference for the first time \\
            k     & if processing the conference that is current in 
                    your conflist \\
            l     & if you are only an observer\index{observer:condition} \\
            m     & if you have mail \\
            n     & if there are new (brandnew or newresponse) items \\
            N     & if \# file is new \\
            O     & if you were previously in another conference this session \\
            r     & if there are newresponse\index{newresponse:condition} 
                    items \\
            s     & if you are a fair-witness\index{fairwitness:condition} \\
            x     & if \# flag is on (1=report header, 2=line in body, 
                    4=report summary) \\
            y     & if there are unseen\index{unseen:condition} items \\
         %}
         \end{supertabular}
\vspace{12pt}

         File numbers for separators which access file numbers for a 
         conference are:


         \begin{tabular}{ll}
         %{
            File Number & Description \\ \hline

            0           & login \\
            1           & logout \\
            2           & index \\
            3           & bulletin \\
            4           & welcome \\
            5           & htmlheader \\
         %}
         \end{tabular}

\vspace{12pt}

      %}

      \subsection{Conference separator variables}
      %{

         \tablefirsthead{Variable & Description \\ \hline}
         \tablehead{\multicolumn{2}{l}{\small\slshape continued from previous page}\\\hline Variable & Description \\ \hline}
         \tabletail{\hline\multicolumn{2}{r}{\small\slshape continued on next page}\\}
         \tablelasttail{\hline}
         \begin{supertabular}{ll}
         %{
            bullmsg  &     output of ``display bulletin''\\
               &\hspace{1cm} Default :``\%(1x\%3g\%)\%c''\\
            checkmsg  &    conference information for check command\\
               & \hspace{1cm} Default :``\%(1x$\backslash$nNew resp 
                 \$\{item\}s\ \ \$$\wedge$\{conference\} name$\backslash$n\%)
                 $\backslash$\\
               & \hspace{1cm}\%(2x\%(k--\%E\ \ \%)\%(B$>$\%E \%)\%4r \%4b
                 \ \ \ \ \%s$\backslash$ \\
               & \hspace{1cm}\%(B\ \ (where you currently are!)\%)\%)''\\
            confindexmsg & format of a conference description line for 
                           index command\\
               & \hspace{1cm} Default :``\%20s.....\%c''\\
            confmsg     &  output of ``display conference''\\
               & \hspace{1cm} Default :``\%Q\$\{conference\} name\ \ \ :
                 \%s $\backslash$n \\
               & \hspace{1cm} Directory\ \ \ \ \ \ \ \ \ : \%d$\backslash$n$
                 \backslash$\\
               & \hspace{1cm} Participation file: \%w/\%p$\backslash$n$
                 \backslash$ \\
               & \hspace{1cm} Security type\ \ \ \ \ : \%t$\backslash$n$
                 \backslash$\\
               & \hspace{1cm}\%(i\%i \$\{item\}\%S numberd \%f-\%l$
                 \backslash$ \\
               & \hspace{1cm}\%ENo \$\{items\} yet.\%)''\\
            edbprompt  &   prompt to confirm response entry\\
               & \hspace{1cm} Default :``Ok to enter this response? ''\\
                 groupindexmsg &format of a group header line for index command\\
               & \hspace{1cm} Default :``\ \ \ \ \ \ \ \ \ \ \ \ \ \ \ \ \ \ \ \ **\%C**''\\
            indxmsg &      output of ``display index''\\
               & \hspace{1cm} Default :``\%(1x\%2g\%)\%c''\\
            joinmsg &      output of ``join'' with no arguments\\
               & \hspace{1cm} Default :``Join: which \$\{conference\}?
                 $\backslash$n$\backslash$\\
               & \%QYou are currently in the \%s \$\{conference\}.''\\
            joqprompt &    prompt when deciding to become a member of a 
                           conference\\
               & \hspace{1cm} Default :``$\backslash$nJoin, quit, or help? ''\\
            linmsg  &      output when entering a conference\\
               & \hspace{1cm} Default :``\%(1x\%(2x$\backslash$n\%)
                 \%0g\%)\%(2x\%(3N\%3g\%)$\backslash$\\
               & \hspace{1cm}\%(1x$\backslash$n\%)\%(n\%(r\%r newresponse \$
                 \{item\}\%S\%)$\backslash$\\
               & \hspace{1cm}\%(b\%(r and \%)\%b brandnew \$\{item\}\%S\%)
                 $\backslash$n\%)$\backslash$\\
               & \hspace{1cm}\%(i\%i \$\{item\}\%S numbered \%f-\%l$\backslash$n
                 \%)$\backslash$\\
               & \hspace{1cm}\%(sYou are a \$\{fairwitness\} in this 
                 \$\{conference\}.$\backslash$n\%)$\backslash$\\
               & \hspace{1cm}\%(lYou are an observer of this \$\{conference\}.
                 $\backslash$n\%)$\backslash$\\
               & \hspace{1cm}\%(jYour name is $\backslash$``\%u$\backslash$'' 
                 in this \$\{conference\}. $\backslash$n$\backslash$\\
               & \hspace{1cm}\%($\sim$O\%(4F$\backslash$n\%4g$\backslash$\\
               & \hspace{1cm}\%E$\backslash$n$>>>>$ New users: type HELP 
                 for help.  $\backslash$n\%)\%)\%)\%)\%c''\\
            listmsg  &     conference information for list command\\
               & \hspace{1cm} Default :``\%(1x$\backslash$n\$\{item\}s sec time
                 \ \ \ \ \ \ \ \ \ \ \ \ \ \ \ \ \ \ \ \ \ \$\{conference\}
                 $\backslash$n\%)$\backslash$\\
               & \hspace{1cm}\%(2x\%(B$>$\%E \%)\%4i \%2t\%(sF\%E\%(lR\%E \%)\%) 
                 \%m \%s\%)''\\
            loutmsg  &     output when leaving a conference\\
               & \hspace{1cm} Default :``\%(1x\%1g\%)\%c''\\
            mailmsg  &     announcement when you have mail\\
               & \hspace{1cm} Default :``You have \%(2xmore \%)mail.''\\
            noconfp  &     main Yapp prompt, if not in a conference\\
               & \hspace{1cm} Default :``$\backslash$nType HELP CONFERENCES for 
                 a list of \$\{conferences\}s.$\backslash$n$\backslash$\\
               & \hspace{1cm}Type JOIN $<$NAME$>$ to access a \$\{conference\}.
                 $\backslash$n YAPP: ''\\
            obvprompt &    prompt after each item when reading, if respond 
                           is not allowed\\
               & \hspace{1cm} Default :``$\backslash$n[\%\{curitem\}/\%l] 
                 Can't respond, pass? ''\\
            partmsg   &    person info, for participants command\\
               & \hspace{1cm} Default :``\%(2x\%10v \%o \%u\%)$\backslash$ \\
               & \hspace{1cm}\%(4x$\backslash$n\%k participant\%S total.\%)
                  $\backslash$ \\
               & \hspace{1cm}\%(1x$\backslash$n\ \ \ \ \ login\ \ \ \ \ \ \ \ 
                 \ \ \ last time on\ \ \ \ \ name$\backslash$n\%)'' \\
            printmsg  &    report header, for print command\\
               & \hspace{1cm} Default :``\%QPrinted from \$\{conference\} 
                 \%d (\%s)''\\
            prompt   &     main Yapp prompt, if in a conference\\
               & \hspace{1cm} Default :``$\backslash$nYAPP:\ ''\\
            rfpprompt &    prompt after each item when reading, if respond 
                 is allowed\\
               & \hspace{1cm} Default :``$\backslash$n[\%\{curitem\}/\%l] 
                 Respond, forget, or Pass?\ ''\\
            scribok   &    prompt to verify scribbling\\
               & \hspace{1cm} Default :``Ok to scribble this response? ''\\
            text     &     prompt while entering text (no separators allowed 
                           for this prompt)\\
               & \hspace{1cm} Default :``$>$''\\
            wellmsg  &     output of ``display welcome''\\
               & \hspace{1cm} Default :``\%(1x\%4g\%)\%c''\\
         %}
         \end{supertabular}   
      %}
\vspace{12pt}

      \subsection{Item separators }
      %{

         Item separators apply in only for item separator
         variables.  For any codes which use a number, the number defaults to 0
         unless otherwise stated.  A code is specified with: \%Z\#X 
         where Z is an optional `Z' or `z' meaning print `No' or 
         `no' in place of 0, \# is an optional number, and X is some 
         character below.
   

         \tablefirsthead{Separator & Description \\ \hline}
         \tablehead{\multicolumn{2}{l}{\small\slshape continued from previous page}\\\hline Separator & Description \\ \hline}
         \tabletail{\hline\multicolumn{2}{r}{\small\slshape continued on next page}\\}
         \tablelasttail{\hline}
         \begin{supertabular}{ll}
         %{
            a     & full name of author of current response \\
            C     & short name of current conference \\
            d     & timestamp of current response (default format is 1) \\
            e     & email address of author of current response \\
            h     & subject (header) of current item \\
            i     & current item number \\
            l     & login of author of current response \\
            k     & size in Kbytes of text of current response (rounded up) \\
            L     & text of current line of current response \\
            n     & highest response number in current item \\
            N     & current line number in current response \\
            p     & number of parent response \\
            q     & size in bytes of text of current response \\
            r     & current response number \\
            s     & number of lines of text in current response \\
            t     & timestamp of current response (default format is 0) \\
            u     & uid of author of current response \\
         
         %}
         \end{supertabular}
\vspace{12pt}

      %}

      \subsection{Item conditionals}
      %{

         Item conditionals apply only to item separator
         variables.  For any codes which use a number, the number defaults to 0
         unless otherwise stated.  Codes inside conditionals are used only if
         the condition is true.  A code is specified with: \%(!\#X 
         where ! is an optional `!' or `\~' meaning to negate the condition
         \# is an optional number (as explicitly defined), 
         and X is some character below.
   
         \tablefirsthead{Conditional & Description \\ \hline}
         \tablehead{\multicolumn{2}{l}{\small\slshape continued from previous page}\\\hline Conditional & Description \\ \hline}
         \tabletail{\hline\multicolumn{2}{r}{\small\slshape continued on next page}\\}
         \tablelasttail{\hline}
         \begin{supertabular}{ll}
         %{
            B     & if starting to process another report  \\
            D     & if date flag is on \\
            F     & if numbered flag is on \\
            I     & if starting to process another item \\
            L     & if there is text at current response line \\
            N     & if current response number > 0 \\
            O     & if starting to process another item  \\
            p     & if response is a response to another one \\
            P     & if last number output was not 1 \\
            R     & if starting to process another response \\
            T     & if formfeed flag is on \\
            U     & if uid flag is on    \\
            V     & if current response is censored  \\
            W     & if current response is scribbled \\
            x     & if \# flag is on (1=report header, 2=line in body, 
                    4=report summary) \\
            X     & if current response is retired   \\
            Y     & if current response is forgotten \\
            Z     & if current response is frozen    \\
          %}
          \end{supertabular}
\vspace{12pt}


      %}

      \subsection{Item separator variables}
      %{

      \tablefirsthead{Variable & Description \\ \hline}
      \tablehead{\multicolumn{2}{l}{\small\slshape continued from previous page}\\\hline Variable & Description \\ \hline}
      \tabletail{\hline\multicolumn{2}{r}{\small\slshape continued on next page}\\}
      \tablelasttail{\hline}
      \begin{supertabular}{ll}
      %{
          fsep\index{fsep} &  output of find command \\
             & \hspace{1cm} Default :``\%(R\#\%i.\%r \%a (\%l) 
               $\backslash$n\%)\%7N: \%L'' \\
          isep\index{isep} &  long item header (default for read command) \\
             & \hspace{1cm} Default :``$\backslash$n\$$\wedge$\{item\} 
               \%i entered \%d by \%a (\%l)$\backslash$n \%h''\\
          ishort\index{ishort} & short item header (default for browse command) \\
             & \hspace{1cm} Default :``\%(B\$$\wedge$\{item\} Resps \$$\wedge$
               \{subject\}$\backslash$n$\backslash$n\%)\%4i \%5n \%h'' \\
          nsep\index{nsep} &    new response status, for read command\\
             & \hspace{1cm} Default :``$\backslash$n\%(N\%r new of \%)
               \%zn response\%S total.''\\
          replysep\index{replysep} & response included in reply mail text\\
             & \hspace{1cm} Default :``\%(1xIn \#\%i.\%r of the \%C 
               \$\{conference\}, you write:\%)$\backslash$\\
             & \hspace{1cm}\%(2x$>$ \%L\%)\%(4x\%)''\\
         rsep\index{rsep} & response header for read command\\
             & \hspace{1cm} Default :``$\backslash$n\#\%i.\%r \%a (\%l) '' \\
         txtsep\index{txtsep} &     line of response text in read command \\
             & \hspace{1cm} Default :``\%(L\%(F\%7N:\%)\%X\%L\%)'' \\
         zsep\index{zsep} &       item footer, for read command \\
            & \hspace{1cm} Default :``\%(T$\wedge$L\%)\%c''\\
      %}
      \end{supertabular} 
      %}

      Following are some of the reserved variable names.  Most have default 
      values that can be checked with {\tt display} {\em variable}.  Undefined 
      variables expand to the empty string, rather than generating an error.

%      \begin{table}[h]
%      \caption{WWW read-only variables\label{t:wwwROvars}}
%      \centering
      \subsection*{WWW Read/only variables}
      %{
         \begin{tabular}{lp{4.5in}} 
         Variable Name & Description \\ \hline
         nobody\index{nobody} 
            & login of account which CGI scripts run as \\
         pathinfo\index{pathinfo}   
            & contents of PATH\_INFO environment variable\\
         querystring\index{querystring} 
            & contents of QUERY\_STRING environment variable 
              (note that variables sent via QUERY\_STRING are 
              automatically set for you in Yapp.) \\
         requestmethod\index{requestmethod}
            & contents of REQUEST\_METHOD environment variable\\
         remoteaddr\index{remoteaddr} 
            & contents of REMOTE\_ADDR environment variable\\
         remotehost\index{remotehost} 
            & contents of REMOTE\_HOST environment variable\\
         sysop\index{sysop}   &   login of sysop account\\
         wwwdir\index{wwwdir} &   base directory for Yapp www files\\ \hline
         \end{tabular}
      %}
%      \end{table}
\vspace{12pt}

      \subsection*{Miscellaneous Read/only variables}
      %{
         \tablefirsthead{Variable & Description \\ \hline}
         \tablehead{\multicolumn{2}{l}{\small\slshape continued from previous page}\\\hline Variable & Description \\ \hline}
         \tabletail{\hline\multicolumn{2}{r}{\small\slshape continued on next page}\\}
         \tablelasttail{\hline}
         \begin {supertabular}{ll}
         %{
         address    &  mailing list email address of a maillist~\index{maillist} conference \\
         bbsdir     &  bbs system base directory~\index{bbsdir} \\
         brandnew   &  number of brandnew items~index{brandnew} \\
         cacl       &  access control list for creating new items in this conference~\index{cacl} \\
         canaacl    &  can change access control list~\index{canaacl} \\
         cancacl    &  can create topics in conference according to access control list~\index{cancacl} \\
         canracl    &  can read conference according to access control list ~\index{canracl}\\
         canwacl    &  can respond to items in conference according to access control list~\index{canwacl} \\
         cflist     &  list of conferences in your .cflist \index{cflist}\\
         conflist   &  list of possible conferences \index{conflist} \\
         confname   &  short name of current conference \index{confname} \\
         curitem    &  current item number \index{curitem}\\
         curline    &  current line number of response \index{curline}\\
         curresp    &  current response number \index{curresp}\\
         cursubj    &  current item subject \index{cursubj} \\
         email      &  your email address \index{email}\\
         exit       &  exit status of last Unix command executed \index{exit}\\
         firstitem  &  first item number \index{firstitem}\\
         fromlogin  &  login of current response author \index{fromlogin}\\
         fromname   &  full name of current response author \index{fromname}\\
         fullname   &  your full name in this conference \index{fullname}\\
         fwlist     &  list of fairwitnesses in this conference \index{fwlist}\\
         highresp   &  highest response number in range specified \index{highresp}\\
         home       &  your home directory \index{home}\\
         hostname   &  hostname of local system on which Yapp is running \index{hostname}\\
         isbrandnew &  is the current item brandnew \index{isbrandnew}\\ 
         isnew      &  is the current item brandnew or contain a new response \index{isnew}\\ 
         isnewresp  &  does the current item contain a new response \index{isnewresp}\\ 
         lastitem   &  last item number \index{lastitem}\\
         lastresp   &  number of responses in current item \index{lastresp}\\
         login      &  your login \index{login}\\
         lowresp    &  lowest response number in range specified \index{lowresp}\\
         mode       &  current prompt index \index{mode}\\
         newresp    &  number of newresponse items \index{newresp}\\
         nextconf   &  next conference in your .cflist with new items \index{nextconf}\\
         nextitem   &  next item in specified range \index{nextitem}\\
         numitems   &  number of items \index{numitems}\\
         partdir    &  directory containing your participation files and .cflist \\
         pid        &  current Yapp process id \index{pid}\\
         prevconf   &  previous conference in your .cflist with new items \index{prevconf}\\
         previtem   &  previous item in specified range \index{previtem}\\
         racl       &  access control list for joining and reading current conference \index{racl}\\
         seenresp   &  highest response number seen \index{seenresp}\\
         status     &  current internal status flags \index{status}\\
         uid        &  your uid \index{uid}\\
         unseen     &  number of unseen items\index{unseen} \\
         wacl       &  access control list for writing responses in current conference \index{wacl}\\
         work       &  your Yapp work directory \index{work}\\
         %
         \end{supertabular}
      %}
\vspace {12pt}

      \subsection*{Extra variables imported from environment variables}
      %{

         \begin{tabular}{ll}
         Variable Name & Description \\ \hline
         alpha         & Value is value of ALPHA environment variable\\
         beta          & Value is value of BETA environment variable\\
         gamma         & Value is value of GAMMA environment variable\\
         delta         & Value is value of DELTA environment variable\\ \hline
         \end{tabular}

         \vspace{12pt}

         For more information on separators and conditionals see the entries 
         on {\tt miscseps}, {\tt confseps}, {\tt itemseps}, {\tt miscconds},
         {\tt confconds}, {\tt itemconds}, {\tt fileseps}, and {\tt dateseps} 
         in the Yapp Manual.
      %}
   %}

   \section{Customizing WWW output and options} \label{s:sysconfig}
   %{
      An rc file\index{rc files!web} containing global commands and 
      definitions for the WWW is located in {\em bbsdir}/www/rc.  It contains 
      the default WWW settings. 
   
      If you would like to customize these defaults you may change them by 
      accessing the {\em Configure System}\index{configure system page} link 
      from the Main Menu page while logged in as {\em sysop}\index{sysop}.  
      The modifications will be stored in the rc.{\em yapp-bin} file, where 
      {\em yapp-bin} is the directory alias used by httpd.  See 
      chapter~\ref{c:start} on Getting Started for more on information 
      setting up the directory alias.

      For more information on the separators used in the default macros, 
      see section~\ref{s:sep} on Separators and Macros.

      \subsection{Using the Configure System page}
      %{

      The Configure system page allows the {\em sysop} to configure the system 
      for the WWW.  From this page you can change the following types of items: 
      System Identification, Configuration Options, Terms, Page Format, 
      Text Format, Return Buttons~\index{buttons}, Other Buttons, and 
      HTML Tags in Responses.

      \subsubsection*{SYSTEM IDENTIFICATION}
  
         \begin{description}
         \item[Local Host Name in URLs] 
         \mbox{}\newline
            This is your WWW host name to be included in URLs.

         \item[Local HTTP port in URLs] 
         \mbox{}\newline
            Port number to include in URLs for your host.

         \item[URL of welcome page]
         \mbox{}\newline
            This is the URL of the page you wish users to see when they follow 
            the ``Return to welcome page'' link.
         \end{description}

      \subsubsection*{TERMS}
       
         Here, you can redefine some of the terms used in messages and on 
         WWW pages to be more understandable to your users.

         You can redefine the terms:
         \vspace{12pt}

         \begin{tabular}{ll}
            Variable    & Default Value\\ \hline
            conference\index{conference}   &   conference\\
            fairwitness\index{fairwitness} &   fairwitness\\
            item\index{item}               &   item\\
            subject\index{subject}         &   subject\\ \hline
         \end{tabular}

         \vspace{12pt}
         See section~\ref{s:var} on Yapp configuration variables for more 
         information on these terms.

      \subsubsection*{CONFIGURE OPTIONS}

         {\bf Do you wish users to be able to enter responses using a 
          pseudonym? }
         \par
         If you do not allow pseudonyms, the user's full name in the conference
         will always be used when entering responses from the web.

      \subsubsection*{PAGE FORMAT}

         Each web page is essentially divided into three pieces: the header,
         the body, and the footer.  Here you can define:
         \vspace{12pt}

         \begin{tabular}{lll}
            & Variable Name & Description \\ \hline
            & header\index{header} & Standard page header \\
            & footer\index{footer} & Standard page footer \\ \hline
         \end {tabular}

         \vspace{12pt}
         To customize the body of a page, you will need to edit the template
         associated with the page.  The templates are located in the
         {\em bbsdir}/www/templates directory.  See section~\ref{s:template}
         for more information on the templates.

      \subsubsection*{TEXT FORMAT}
      %{

         This is where you define text formats for specific pages.

         \begin{tabular}{lll}
         & Variable Name & Description \\ \hline
         & ishort\index{ishort} & Format of the list of topics inside a conference \\
         && - Changes appearance on a Conference Welcome Page \\

         & isep\index{isep} & Format of a topic header \\
         &&            - Changes appearance on a Read Item Page\\

         & rsep\index{rsep} & Format of a response header \\
         &&         - Changes appearance on a Read Item Page \\

         & censored\index{censored} & Format of a censored notice \\
         &&      - Changes appearance on a Read Item Page \\

         & scribbled\index{scribbled} & Format of a scribbled notice  \\
         &&   - Changes appearance on a Read Item Page\\ \hline
         \end{tabular}

         \vspace{12pt}
         {\em NOTE: A Conference Welcome Page is the first page you come to when
            entering a conference.  Normally the contents of the conference 
            htmlheader file are displayed at the top of the page. 

            A Read Item Page, is the page you get when you follow the link to 
            any of the items/topics on a Conference Welcome Page.
         }
      %}

      \subsubsection*{RETURN BUTTONS}
         \index{buttons}
         Here you define the buttons which can be used to return to other 
         pages.

         \vspace{12pt}
         \begin{tabular}{lll}
         &Variable Name & Description \\ \hline
         & confreturn\index{confreturn} & Return to the \_\_\_\_\_ conference\\
         & mainreturn\index{mainreturn} & Return to the main menu \\
         & welcreturn\index{welcreturn} & Return to the welcome screen \\ \hline
         \end{tabular}

      \subsubsection*{OTHER BUTTONS}
         \index{buttons}
         Here is where you redefine the buttons such as:

         \vspace{12pt}
         \begin{tabular}{lll}
         & Variable Name & Description \\ \hline
         & help\index{help} & Help Button \\
         & cnext\index{cnext} & Next conference \\
         & cprev\index{cprev} & Previous conference \\
         & znext\index{znext} & Next topic  \\
         & zprev\index{zprev} & Previous topic \\
         & enter\index{enter} & Enter new topic \\
         & zfreeze\index{zfreeze} & Freeze/Thaw \\
         & zforget\index{zforget} & Forget/Remember \\
         & kill\index{kill} & Kill item button \\
         & zretire\index{zretire} & Retire item button \\ 
         & retitle\index{retitle} & Change subject \\
         & zcensor\index{zcensor} & Censor \\
         & zuncensor\index{zuncensor} & Uncensor \\
         & zscribble\index{zscribble} & Scribble \\ \hline
         \end{tabular}

      \subsubsection*{HTML TAGS IN RESPONSES}
         \index{html tags}
         Here is where you can restrict which HTML tags you want users to be
         allowed to use, and on which tags you want sanity checking. Users will
         not be able to use any HTML tags which you list as illegal, and 
         sanity checking will be done on the tags which you specify must be 
         matched.
      %}

      \subsection{Modifying templates}
      %{
      \index{templates}\label{s:template}
      In addition to the standard WWW marcos, you can also change
      templates to determine  what is displayed for each form and
      on each page.  Shown below is short description of when each 
      of the  templates is used, and default actions upon sucessful or
      unsuccessful completion of Yapp commands.
      \vspace{12pt}
      \par
      \tablefirsthead{&Template Name & Description of Template\\ \hline}
      \tablehead{\multicolumn{2}{l}{\small\slshape continued from previous page}\\\hline &Template Name & Description \\ \hline}
      \tabletail{\hline\multicolumn{2}{r}{\small\slshape continued on next page}\\}
      \tablelasttail{\hline}
      \begin{supertabular}{lll}
      %{
         & browse\index{browse}
           & Display list of topics in a conference, change name in conference,
            etc \\ 
         & browse\_ok\index{browse\_ok}
           & Results of browse (go to browse screen)\\
         & cfadd\index{browse}
           & Results of cfadd (go to createlist screen) \\
         & cfconfig\index{cfconfig}
           & Sysop configuration parameters \\
         & cfcreate\index{cfcreate}
           & Form to collect information to create a conference\\
         & cfcreate\_ok\index{cfcreate\_ok}
           & Results of cfcreate \\
         & cfonce\index{cfonce}
           & Change information in users cfonce file (go to userinfo) \\
         & change\_get\index{change\_get}
           & The initial file change form \\
         & change\_invalid\index{change\_invalid}     
           & When response contains invalid text, let user edit \\
         & change\_ok\index{change\_ok}
           & Results of change command (go to browse screen) \\
         & changelist\_get\index{changelist\_get}     
           & The initial .cflist change form \\
         & changelist\_ok\index{changelist\_ok}
           & Results of changelist command (go to browse screen) \\
         & chfn\_get\index{chfn\_get}
           & Fill out form for changing fullname/email address \\
         & chfn\_ok\index{chfn\_ok}
           & Results of changing fullname/email address \\
         & conf\_timeout\index{conf\_timeout}
           & A timeout screen with "Return to ... conference" link \\
         & createlist\index{createlist}
           & Create a conference list \\
         & edit\_get\index{edit\_get}
           & Edit a responce form \\
         & edit\_invalid\index{edit\_invalid}
           & When invalid text is entered from edit, let them edit again \\
         & enter\_get\index{enter\_get}
           & The initial item entry form \\
         & enter\_invalid\index{enter\_invalid}
           & When item contains invalid text, let user edit \\
         & enter\_ok\index{enter\_ok}
           & Results of enter command (go to browse screen) \\
         & error\index{error}
           & Generic screen displayed when an error occurs \\
         & find\index{find}
           & Form to enter a keyword search \\
         & find\_ok\index{find\_ok}
           & Results of a keyword search \\
         & forget\index{forget}
           & Results of forget command (go to browse screen) \\
         & freeze\index{freeze}
           & Results of freeze command (go to browse screen) \\
         & index\index{index}
           & List of conferences and create conference command \\
         & kill\index{kill}
           & Results of kill command (go to browse screen) \\
         & list\index{list}
           & List of all conferences \\
         & main\index{main}
           & The main menu \\
         & main\_timeout\index{main\_timeout}
           & A timeout screen with "Return to main menu" link \\
         & newuser\_get\index{newuser\_get}
           & Step 1 of newuser registration (select a login) \\
         & newuser\_get2\index{newuser\_get2}
           & Step 2 of newuser registration (enter user info) \\
         & newuser\_ok\index{newuser\_ok}
           & When a user successfully registers a new login \\
         & newuser\_unix\index{newuser\_unix}
           & Display when a user chooses an existing Unix login \\
         & participants\index{participants}
           & Display all participants in a conference \\
         & passwd\_get\index{passwd\_get} 
           & Fill-out form for changing password \\
         & passwd\_ok\index{passwd\_ok}
           & When password is successfully changed \\
         & preserve\index{preserve} 
           & Result of preserve (go to browse screen) \\
         & public\_browse\index{public\_browse}
           & Display list of topics in public readonly mode \\
         & public\_find\index{public\_find} 
           & Form to enter a key word search when not authenticated \\
         & public\_main\index{public\_main}
           & The main menu when not authenticated \\
         & public\_read\_footer\index{public\_read\_footer}
           & Item footer in public readonly mode \\
         & public\_read\_header\index{public\_read\_header} 
           & Item header in public readonly mode \\
         & read\_footer\index{read\_footer}        
           & Item footer in a normal conference \\
         & read\_header\index{read\_header}        
           & Item header in a normal conference \\
         & remember\index{remember}
           & Results of remember command (go to browse screen) \\
         & respond\_get\index{respond\_get}
           & The initial response form \\
         & respond\_invalid\index{respond\_invalid}
           & When response contains invalid text, let user edit \\
         & respond\_ok\index{respond\_ok}
           & Results of respond command (go to browse screen) \\
         & retire\index{retire}
           & Results of retire command (go to browse screen) \\
         & retitle\_get\index{retitle\_get}
           & Form to change the title of an item \\
         & sysconfig\index{sysconfig}
           & WWW configuration page for Sysop \\
         & sysconfig\_ok\index{sysconfig\_ok}
           & Result of sysconfig changes (go to main menu) \\
         & sysop\index{sysop}
           &  Sysop Menu \\
         & thaw\index{thaw}
           & Results of thaw command (go to browse screen) \\
         & unretire\index{unretire}           
           & Results of unretire command (go to browse screen) \\
         & userinfo\index{userinfo}
           & Display User Info screen \\
      %}
      \end{supertabular}

      \vspace{12pt}

      Modifying the templates will cause system wide changes, not 
      conference specific changes.  Feel free to experiment with the
      templates to create the enviornment you feel will best meet the
      needs of your system.  

      Standard separators and macros are imbedded in the templates by default.
      The standard WWW macros can be redefined from the  System Configuration 
      page. Often the macros are the only thing that will need to be 
      updated to make your page appear the way that you wish.  See 
      section~\ref{s:sysconfig} for more information on changing these 
      macros.  

      Although the templates are system specfic, the macros can be defined
      within each conference, to give each conference a unique feel.  
      Any changes for a specific conference need to be made to the conference
      rc file.  By default this file is empty, so it will require some work
      to redefine any macros.
      %}
   %}

   \section{Customizing output from Unix and global options} \label{s:rc}
   %{
      The most basic of the rc files\index{rc files!Yapp} is the Yapp
      rc file {\em bbsdir}/rc. This file will be executed whenever Yapp is 
      run.  The definitions in this file can be overridden by any other rc 
      files which are executed. 

      The Yapp rc file should contain commands that will be executed by 
      everyone when Yapp is started.  In general, these are to define global 
      aliases and customize output.  See the section on ``define'' in the Yapp 
      Manual for more information on the {\tt define} comand.

      \qquad Example: \\
      \qquad def editor 258 vi \\
      \qquad def pager more \\
      \qquad def helpcmd 'help commands' \\
      \qquad def who       9   'unix who \\
      \qquad def write     9   'unix ``/usr/local/bin/write'' \\
      \qquad def talk      9   'unix ``/usr/bin/talk'' \\
      \qquad def dir\_ectory 9 'unix ``/bin/ls'' \\
      \qquad def type  9 'unix ``/bin/cat'' \\
      \qquad def pwd   9 'unix ``/bin/pwd'' \\
      \qquad def chat  9 'unix ``/usr/local/bin/write'' \\
      \qquad def shell 9 'unix ``/bin/csh'' \\
      \qquad def noconfp ``Type HELP CONFERENCES for a list of conferences
$\backslash$n$\backslash$\\
Type JOIN $<$NAME$>$ to access a conference.$\backslash$n$\backslash$n$>$ \%c'' \\

      Each user with a Unix account can create their own aliases and customized 
      output. For more information on the rc files that a user can create, 
      see the section in the Yapp manual for {\em cfonce}\index{cfonce} and 
      {\em cfrc}\index{cfrc}.

      You can also redefine separators in the rc file.  See section~\ref{s:sep}
      on seperators for a description of the Item Seperators you can define 
      in the rc file.

      The following is an example definition of isep.

\qquad define isep ``$\backslash$nConference:\%{confname}$\backslash$nResponse: \%\{curitem\}.\%\{curresp\}$\backslash$nItem$\backslash$ \\
\qquad \%\{curitem\} was entered \%d by \%a  (\%\{fromlogin\}) $\backslash$nSubject: \%\{cursubj\}$\backslash$n''

      \subsection{Changing the Yapp default settings} \label{s:var}
      %{
      \index{default settings}
      The following are the Yapp default settings.  To change a
      default setting globally, you should redefine the associated variable
      in the Yapp rc file ({\em bbsdir}/rc).  To check the values from within 
      Yapp, type {\tt display} {\em variable}.  Undefined variables expand to 
      the empty string, rather than generating an error.  See 
      section~\ref{s:rc} on rc files for more information on globally 
      redefining variables.

      \tablefirsthead{Variable & Default & Description \\ \hline}
      \tablehead{\multicolumn{3}{l}{\small\slshape continued from previous page}\\\hline Variable & Default & Description \\ \hline}
      \tabletail{\hline\multicolumn{3}{r}{\small\slshape continued on next page}\\}
      \tablelasttail{\hline}
      \begin{supertabular}{lll}
      %{
      bufdel\index{bufdel}           & ``;''   
             & buffer delimiter (only first character is significant)\\
      cmddel\index{cmddel}           & ``;''   
             & command delimiter (only first character is significant)\\
      conference\index{conference}   & ``conference'' 
             & what to call a conference/forum/etc\\
      editor\index{editor}           & ``/bin/ed''    
             & Unix command to invoke editor \\
      escape\index{escape}           & ``:''          
             & escape character in text entry mode (first char significant)\\
      fairwitness\index{fairwitness} & ``fairwitness'' 
             & what to call a fairwitness/host/etc\\
      gecos\index{gecos}             & ``,''           
             & GECOS separator in password file (first character significant)\\
      helpcmd\index{helpcmd}         & ``''            
             & default command to execute at Ok prompt when user hits return\\
      item\index{item}               & ``item''        
             & what to call an item/topic/etc\\
      mailbox\index{mailbox}         & ``''            
             & filename to use as your mailbox\\
      pager\index{pager}             & ``/usr/bin/more''
             & Unix command to invoke pager\\
      shell\index{shell}             & ``/bin/sh''      
             & Unix command to invoke as the shell\\
      subject\index{subject}         & ``subject''      
             & what to call the subject/title/etc of an item\\
      visual\index{visual}           & ``''             
             & Unix command to invoke visual editor\\
      censored\index{censored}       & ``\ \ \ $<$censored$>$''
             & response is censored notice\\
      scribbled\index{scribbled}     & ``\ \ \ $<$censored \& scribbled$>$'' 
             & response is scribbled notice\\
      %}
      \end{supertabular}

      For more information on variables, see the entry on variables in the 
      Yapp Manual.
      %}
   %}

%}

\chapter{Special Cases} \label{c:specialcases}
%{
   \section{Running multiple virtual systems on the same machine}
   %{
       \index{virtual systems}
       
       Yapp allows for multple virtual systems to be run on the
       same machine.  The virtual systems use the same conflist, desclist,
       participation files, and other Yapp files.  Virtual systems can be
       used to make your system appear differently depending upon how
       it was accessed.  Each virtual systems can have a different set of 
       default information and templates.

       A virtual system can be very useful if you wish to experiment with
       the global macros and templates, but you don't want to impact your
       current users.

       To create a virtual system, you need to add another set of information
       to your httpd files.

       For example if you were to add

       \begin{verbatim}
       ScriptAlias /test-bin/ {\em bbsdir}/www/cgi-bin/
       \end{verbatim}

       to your NCSA or Apache HTTP srm.conf you would have a virtual test
       system which you could access a Sysop page to reconfigure without
       upsetting your regular system.
   %}
%}

\chapter{Yapp Files}
%{
   \section{Overview of Important Files}
   %{
      Following is a list of all important files used by YAPP.  You can get more
      info on the format of a file with "help file $<filename>$", where 
      $<filename>$ is the string in the right column below.

      \subsection{Yapp General Files}
      %{

         \tablefirsthead{Parameter & Description & Default \\ \hline}
         \tablehead{\multicolumn{3}{l}{\small\slshape continued from previous page}\\\hline File & Description &  See Yapp Manual: \\ \hline}
         \tabletail{\hline\multicolumn{3}{r}{\small\slshape continued on next page}\\}
         \tablelasttail{\hline}
         \begin{supertabular}{lp{7cm}l}
         %{
            /etc/yapp.conf\index{files!yapp.conf}  
                            & local YAPP configuration information               
                            & yappconf \\
            {\em bbsdir}/censored\index{files!censored}    
                            & log of all censoring and scribbling                
                            & censored \\
            {\em bbsdir}/conflist\index{files!conflist}    
                            & mapping from conference names to directories       
                            & conflist \\
            {\em bbsdir}/desclist\index{files!desclist}    
                            & mapping from conference names to descriptions     
                            & desclist \\
            {\em bbsdir}/errorlog\index{files!errorlog}    
                            & log of all YAPP errors encountered                 
                            & errorlog \\
            {\em bbsdir}/maillist\index{files!maillist}    
                            & mapping from email addresses to conference names   
                            & maillist \\
            {\em bbsdir}/rc\index{files!rc}          
                            & commands to execute when starting bbs              
                            & bbsrc \\
            {\em licensedir}/registered\index{files!registered}  
                            & license information                                
                            & registered \\
                  
         %}
         \end{supertabular}
      %}

      \subsection{Conference files}
      %{

         \tablefirsthead{Parameter & Description & Default \\ \hline}
         \tablehead{\multicolumn{3}{l}{\small\slshape continued from previous page}\\\hline File & Description &  See Yapp Manual: \\ \hline}
         \tabletail{\hline\multicolumn{3}{r}{\small\slshape continued on next page}\\}
         \tablelasttail{\hline}
         \begin{supertabular}{lp{7cm}l}
         %{
   
            {\em confdir}/\_\#\index{files!\_\#}         
                             & an item file
                             & item \\
            {\em confdir}/acl\index{files!acl}        
                             & access control list
                             &acl \\
            {\em confdir}/article\index{files!article}    
                             & highest article number in a nresgroup conference 
                             &article \\
            {\em confdir}/bulletin\index{files!bulletin}   
                             & shown when entering conference, if file is new
                             &bulletin \\
            {\em confdir}/config\index{files!config}     
                             & conference configuration file          
                             &config \\
            {\em confdir}/htmlheader\index{files!htmlheader} 
                             & conference header displayed from WWW
                             & htmlheader \\
            {\em confdir}/index\index{files!index}      
                             & displayed with "display index" command
                             & index \\
            {\em confdir}/log\index{files!log}        
                             & kill/link/freeze/thaw/retire/unretire log
                             & log \\
            {\em confdir}/login\index{files!login}      
                             & displayed when entering conference
                             & login \\
            {\em confdir}/logout\index{files!logout}     
                             & displayed when leaving conference
                             & logout \\
            {\em confdir}/observers\index{files!observers}  
                             & optional user list to be referenced by acl file
                             & observers  \\
            {\em confdir}/originlist\index{files!originlist} 
                             & optional list of hosts/addresses allowed to join 
                             & originlist \\
            {\em confdir}/rc\index{files!rc}         
                             & commands to execute when entering
                             & rc \\
            {\em confdir}/secret\index{files!secret}     
                             & conference password (if any)
                             & secret \\
            {\em confdir}/sum\index{files!sum}        
                             & conference information summary file
                             & summary \\
            {\em confdir}/ulist\index{files!ulist}      
                             & list of participants in the conference
                             & ulist \\
            {\em confdir}/welcome\index{files!ulist}
                             & shown when joining conference for the first time 
                             & welcome \\
          %}
          \end{supertabular}
     %}

     \subsection{User files}
     %{
         \tablefirsthead{File & Description & See Yapp Manual \\ \hline}
         \tablehead{\multicolumn{3}{l}{\small\slshape continued from previous page}\\\hline File & Description &  See Yapp Manual: \\ \hline}
         \tabletail{\hline\multicolumn{3}{r}{\small\slshape continued on next page}\\}
         \tablelasttail{\hline}
         \begin{supertabular}{lp{7cm}l}
         %{
   
            {\em homedir}/.cfdir/    
                        & if exists, this is {\em workdir}.  Otherwise 
                         {\em workdir} = {\em homedir}
                        & cfdir \\
            {\em homedir}/.cfjoin    
                        & commands to execute when joining a new conference  
                        & cfjoin \\
            {\em homedir}/.cflist    
                        & list of conferences for check \& next commands      
                        & cflist \\
            {\em homedir}/.cfonce    
                        & commands to execute when starting bbs              
                        & cfonce \\
            {\em homedir}/.cfrc      
                        & commands to execute when entering a conference     
                        & cfrc \\
            {\em homedir}/.CONF.cf   
                        & conference participation file                      
                        & partfile \\
         %}
         \end{supertabular}
      %}
   %}

   \section{Log Files} \label{c:logfiles}
   %{
      \subsection{The error log}
      %{
         The error log\index{errorlog} is located in the {\em bbsdir} directory, 
         and contains a list of the Yapp errors.  This file may need to 
         be removed periodically since it can grow without bound.  If you are
         experiencing any problems with Yapp this file may contain the information
         you need to diagnose the problem.
      %}
   
      \subsection{The usage log}
      %{
         \index{usagelog}
         The usage log is located in the {\em bbsdir} directory, and  contains 
         a history of the number of times Yapp was accessed from the WWW 
         on a given day.  The entry is can be read as:
         \par
         {\tt Date: Number of Accesses}
         \par
        
         For example:
         {\tt
         \begin{verbatim}
   Mon Sep 11: 34
   Tue Sep 12: 97
   Wed Sep 13: 90
   Thu Sep 14: 79
   Fri Sep 15: 37
         \end{verbatim}
         }
   
         Monitor this log if you are worried about exceeding the
         maximum number of WWW hits per day alloted with your license.  You
         can get the current number of hits from the debug pages.  See 
         section~\ref{s:debug} for more information on the 
         debug~\index{debug} pages.  The useage log\index{usagelog} file 
         can grow without bound, and should probably be deleted periodically 
         by a system administrator.
   
      %}
   
      \subsection{The censor log}
      %{
         \index{censored}
         All censored, scribbled, and edited text is appended to 
         {\em bbsdir}/censored along with a record of by whom and when.  This 
         file can grow without bound, and should probably be deleted periodically 
         by a system administrator.
   
         For more information, see the Yapp Manual entries  on 
         {\tt censor}, {\tt scribble}, and {\tt edit}.
      %}
   
      \subsection{Logging other events}
      %{
         \index{log}
         In addition to error logging, the following other events can
         be logged:
   
   
         {
         \begin{tabular}{lll}
            & Event Name                     &  Description \\ \hline
            & censor  \index{censor!log}     &  A response was censored\\
            & cfcreate\index{cfcreate!log}   &  A new conference was created \\
            & cfdelete\index{cfdelete!log}   &  A conference was deleted \\
            & edit    \index{edit!log}       &  A response was edited \\
            & enter   \index{enter!log}      &  A new item was entered \\
            & freeze  \index{freeze!log}     &  An item was frozen \\
            & join    \index{join!log}       &  A user entered a conference \\
            & kill    \index{kill!log}       &  An item was killed \\
            & leave   \index{leave!log}      &  A user left a conference \\
            & newjoin \index{newjoin!log}    &  A user joined a conference for the first time \\
            & resign  \index{resign!log}     &  A user resigned membership in a conference \\
      	    & respond \index{respond!log}    &  A response was entered \\
            & retire  \index{retire!log}     &  An item was retired \\
            & retitle \index{retitle!log}    &  The subject of an item was changed \\
            & scribble\index{scribble!log}   &  A response was scribbled \\
            & thaw    \index{thaw!log}       &  An item was thawed \\
            & unretire\index{unretire!log}   &  An item was unretired \\
         \end{tabular}
         }
      \vspace{12pt}


      The message form and content, as well as a log file name for each 
         event are specified in the Yapp rc\index{rc files!Yapp} file. 
         The syntax for constructing a log line in the Yapp rc file is :
   
     {\hspace{20pt} {\tt log} {\em event} {\tt filename} {\tt sepstring} }
        
         The default entries in  Yapp rc file are as follows:
   
   {
     \tt
     \begin{verbatim}
     log thaw     '%{confdir}/log' '%D %{login} thawed %{item} %{curitem}'
     log retire   '%{confdir}/log' '%D %{login} retired %{item} %{curitem}'
     log unretire '%{confdir}/log' '%D %{login} unretired %{item} %{curitem}'
     log kill     '%{confdir}/log' '%D %{login} killed %{item} %{curitem}'
     log retitle  '%{confdir}/log' '%D %{login} retitled %{item} %{curitem}'
     log enter    '%{confdir}/log' '%D %{login} entered %{item} %{curitem}'
     log edit     ''               ''
     log respond  ''               ''
     log censor   ''               ''
     log scribble ''               ''
     log join     ''               '%D %{login} joined %{confname}'
     log leave    ''               '%D %{login} left %{confname}'
     log resign   ''               '%D %{login} resigned from %{confname}'
     log newjoin  ''               '%D %{login} newly joined %{confname}'
     log cfcreate ''               '%D %{login} created %{confname}'
     log cfdelete ''               '%D %{login} deleted %{confname}'
   \end{verbatim}
   }
   
         \vspace{12pt}
         If either the {\tt sepstring} or the {\tt filename} are empty the 
         {\em event} will not be logged. Only the events freeze, thaw, retire, 
         unretire, kill, retitle, and enter are logged by default. 
         All of these events are recorded in a log file which 
         is located in the appropriate conference directory.
   
         Each event can only have one entry in the Yapp rc file.  If you 
         try to create multiple definitions for an event you will get
         the following error:
   
         {\centering Can't redefine constant '{\em event}log'   }\\
         {\centering Can't redefine constant '{\em event}logsep'}\\
   
     
   
         An alternate format for specifying the log {\tt filename} or 
         {\tt sepstring} for a particular {\em event} is:
   
         {\centering constant {\em event}log    {\tt filename}}\\
         {\centering constant {\em event}logsep {\tt sepstring}}
   
         The {\em constant} declaration means that once the value has
         been asigned it can not be changed.  For more information on 
         logging events, see the Yapp Manual entries  on 
         {\tt file bbsrc\index{bbsrc}}, {\tt constant\index{constant}}, 
         and {\tt log\index{log}}.
      %}
   %}
%}
   
\chapter{Troubleshooting} \label {c:trouble}
%{
   \section{Resolving license-related error messages}
   %{
      There are a number of error messages associated with the Yapp 
      license\index{license}:
      \begin{description}
      %{
         \item[License registration information not found]
            \mbox{}\newline
            You need to install the license you received from Armidale Software
            in the Yapp license directory ({\em bbsdir}/license/).  The license 
            file should be {\em bbsdir}/license/registered\index{registered}.  
            It needs to be 
            owned by the Yapp administrator ({\em cfadm}) and mode 644.

         \item[Got error 13 (permission denied) in opening /usr/bbs/license/registered]
            \mbox{}\newline\vspace{-12pt}
            \begin{enumerate}
            \item Make sure that the bbs binary is owned by the conference 
               administrator ({\em cfadm}) and that it is mode 4711.

            \item Make sure that {\em bbsdir}/license/registered is mode 644 
               and owned by the Yapp administrator ({\em cfadm}).
            \end{enumerate}

         \item[Invalid Checksum]
            \mbox{}\newline
            Make sure your license file has the four lines which were mailed 
            to you with no information or blank lines preceeding the 
            ``Registered to:'' line.

         \item[There are already {\em xxx} copies being used. Try again later.]
            \mbox{}\newline
            You have exceeded the number of simultaneous unix users which are 
            upgrading to a larger license.

         \item[The Yapp license limit on hits/day has been exceeded. Try again
               in {\em hh} hours {\em mm} minutes.]
            \mbox{}\newline
            You have exceeded the number of hits from the WWW allowed by your 
            license. You should consider upgrading to a larger license.
      %}
      \end{description}
   %}

   \section{Troubleshooting web page access problems}
   %{
      {\em To be filled in.}

% XXX make sure we talk about setuid scripts here

      \subsection{Using the debug pages} \label {s:debug}
      %{
         There are two debug\index{debug} pages for Yapp, one is in the public 
         directory, and the other is in the restricted directory.  There are 
         links to these pages from the System Configuration page.  You may be 
         able to access the debug pages even if you cannot get to other Yapp 
         pages.  If you need to get to the debug pages, and you cannot get to 
         the System Configuration page, the URLs are:
         \begin{verbatim}
         http://localhost/yapp-bin/public/debug
         http://localhost/yapp-bin/restricted/debug
         \end{verbatim}
         \vspace{-12pt}
   
         These files contain the  version of Yapp you are running, and the 
         version of the scripts you are using.  They also contain process 
         information, host information, user information, browser information, 
         conference information, the extra environment variables, the last 10 
         lines of the Yapp error log, and the number of web hits per day for 
         the last 10 days.


         
      %}

      \subsection{The server encountered an internal error or misconfiguration...}
      %{
         There are many possible causes for the ``internal error or 
         misconfiguration'' error message from the WWW.  The most frequent
         cause of the error is when the bbs executable file is not owned
         by {\em cfadm} or is not running setuid.

         You might get more information by viewing the source of the page 
         you tried to access from the appropriate debug page.  
         You can view the source of pages which do not require authentication 
         from 

{\tt http://localhost/yapp-bin/public/debug } 
         and the source of pages requiring authentication from  

         {\tt http://localhost/yapp-bin/restricted/debug }

         \subsubsection{Yapp is not running setuid}
         %{
            To determine if your bbs executable file is correct, first 
            locate the directory (probabaly /usr/local/bin) where the
            executable is found.
       
         \begin{verbatim}
         $ which bbs
         /usr/local/bin/bbs

         $ ls -l /usr/local/bin/bbs
         -rws--x--x  1 cfadm  bin  532541 Jan 17 17:29 /usr/local/bin/bbs*
         \end{verbatim}

            If your file does not have the premissions as shown in the above
            example you should do:

         \begin{verbatim}
         $ chown cfadm /usr/local/bin/bbs
         $ chmod 4711 /usr/local/bin/bbs
         \end{verbatim}

         %}
         \subsubsection{Missing License or other license errors}
         %{
           \index{license}
            Make sure that you have a license file, and that
            it is installed in the correct directory.  You should have
            a file called {\tt registered} in the directory you choose
            to hold your license.  By default this directory is 
            {\em{bbsdir}/license}.


         \begin{verbatim}
         $ cd /usr/bbs/license 
         $ ls -l registered 
         -rw-r--r--  1 cfadm  wheel  82 Jan 18 04:11 registered
         \end{verbatim}

             If you do not have a license file, or if you are
             having problems with your license contact:
         \begin{verbatim}
         yapp@armidale.ann-arbor.mi.us
         \end{verbatim}

         %}
         \subsubsection{Yapp scripts not setuid for Solaris}
         %{

           If you are running Solaris, make sure that you have
           changed all of the yapp scripts to be owned by 
           {\em cfadm} and mode 4755.  You may need to be
           logged in as root to do this.

         \begin{verbatim}
         $ cd /usr/bbs/www/cgi-bin/restricted
         $ chown cfadm *
         $ chmod 4755 *
         $ cd /usr/bbs/www/cgi-binn/public
         $ chown cfadm *
         $ chmod 4755 *
         \end{verbatim}

         %}

         %{
         %   Should the alias for bbs (yappdebug) be
         %   mentioned in the section?  Change the 
         %   first line of the script that is giving
         %   errors to yappdebug instead of bbs.
         %
         %}

      %}
   %}

   \section{Troubleshooting mailing list conferences}\label{s:tmaillist}
   %{
      If you are having problems with your maillist conferences recieving mail,
      first make sure that Yapp is configured correctly.  Save a piece of mail 
      you have received as a file with all of the header information.  Edit 
      the piece of mail such that the ``To:'' line contains the mailling list 
      address to which the conference is linked.  Make sure that there is a 
      ``From:'' line as well as a ``Date:'' line in the header of the 
      message.  While logged in as {\em cfadm}, send the message to 
      {\tt bbs -i}.

      For example:
      \begin{verbatim}
      $ /usr/local/bin/bbs -i < newmail
      \end{verbatim}
      \vspace{-12pt}

      Check to see if the message appears in the conference linked to the 
      mailing list.

      If the message does {\bf NOT} appear in the conference:
      \begin{enumerate}
         \item Check the maillist file to make sure that you have told Yapp 
            where to send mail which is addressed  to the mailling list.  
            See section~\ref{s:maillist} for more information on the format 
            of the maillist file.

         \item Make sure you were logged in as {\em cfadm} when you sent the 
            message.
	
         \item Make sure that bbs is owned by {\em cfadm} and mode 4711.
      \end{enumerate}

      If the message {\bf DID} appear in the conference, but you are not 
      receiving any mail from the maillist:
      \begin{enumerate}
         \item Make sure there is a line in /etc/aliases which looks 
               like\index{cflink}:\\
            cflink:``\vline \ /usr/local/bin/bbs -i'' \\
	
         \item Make sure you are registered on the mailling list to receive 
            mail.
      \end{enumerate}

      If the mailer-daemon bounces email for a mailing list conference with:\\
      ``unknown mailer error 2''\\
      This typically means that {\tt bbs -i} was not invoked by root, daemon,
      or the Yapp owner ({\em cfadm}).  Make sure that bbs is mode 4711.  

      {\em NOTE: Yapp3.0.10 and earlier versions only accept the 
      -i option if they are executed as root or the Yapp owner({\em cfadm}). 
      Mail aliases will only work with Yapp3.0.11 and higher, since 
      sendmail executes programs (by default) as daemon.}

      If you conference doesn't seem to be sending mail, make sure that the
      conference security type includes the `maillist' flag, and that there 
      is an email address specified on line 6 of the conference configuration
      file. See section \ref{s:create} for more information on creating 
      conferences.
   %}
%}

\chapter{Frequently Asked Question} \label {c:FAQ}
%{
  \begin{enumerate}
   \item {\tt Why does an unlimited license~\index{license} get more support 
       hours than other licenses?}

      The larger (higher-usage) sites usually have a lot more questions 
      and customization they want to do.

   \item {\tt Does "Simultaneous" users refer to "http sessions"?}

      Since Yapp doesn't run in between hits on a web page, http sessions
      usually count as 0. Simultaneous refers to the number of Yapp processes 
      running.

   \item {\tt Why do I feel like a beta tester? Should I be asking for an accomodation
        in price, support, or some other form? } 

      There are new features included in the version you have,
      and essentially those caused problems.  New features (such as simpler
      administration) tend to make your life easier in the long run, but
      sometimes have bugs, so essentially yes you have a beta version of
      code.

      If you'd rather stay away from beta releases, that's okay too.
      There's no difference in price either way.


   \item{\tt If I purchase a license~\index{license} and find out later 
        it's too little, can I add users/hits later? Is there a price penalty?}

      Yes you can always upgrade~\index{upgrade}, for only the cost of the 
      difference in licenses.

   \item{\tt Can I buy additional support time if needed? }

      Yes, such support is usually handled by ILC in San Francisco, who do Yapp
      customer support and reselling for us.  Their people are really good, too.

      You can do this either if you've used up any "with support" hours, or
      if you have a "no support" license~\index{license}.

   \item {\tt How do I set it up so that a conference emails all the members 
        of conference with the new content in the conference as it's 
        posted?  And so that they can reply to specific topics by email? }

      At the moment, you have to link the conference to a mailing list.
      Section \ref{s:maillist} shows how to do this.

    \item {\tt Is it possible to select a particular user in a public
         conference, and prevent them from posting responses, but allowin 
         them to read the discussion while others may read and post?  
         In other words, can you put an individual on "read-only" in a given 
         conference? }

      Yes.  In each conference you have the option of creating an 
      observers file, and setting the conference access control list (acl)
      options such that they do not have permission to respond.

   \item {\tt  Which systems are the Yapp conferences linked to?}

      The Yapp conference is intended to be for discussion about Yapp itself,
      and is linked between a number of systems around the country.  It's
      open to anyone running Yapp who wants it.
      Do "finger yapp@umich.edu" and you'll get the list.

   \item {\tt How much customization by conference is possible?}

       Conference customization is not quite as easy as global configuration.
       You can put commands in the conference rc file by going to the 
       bottom of the conference host page and selecting "rc: conference 
       configuration commands", or from inside Yapp in Unix with "change rc".

      Each of the macros defined in the templates can be modified for
      each conference.  Examples include, but are not limited to the
      buttons at the top of the page.  If there is something which
      is not currently a macro but you want to be modifiable by
      conference, create a macro definition for it so that it
      can be modifiable by conference.

      The tricky part is that you have to know what you want to define.
      For example, to specify an alternate gifs directory, you would add:
      {\tt define gifs /new/full/path/name} to the conference rc file.

   \item {\tt What is the problem when topics refuse to age?  They keep on coming up 
        on the new list, despite the fact that there have been no new posts to 
        that topic; on viewing only the topic header is shown. }

      This is probably related to clock troubles and the consequent 
      post-dated posts; they are disappearing as we catch up with them.
    \end{enumerate}

%}

\appendix

\chapter{Man pages} \label{a:manpages}
%{
   \index{webuser}
   \input{webuser.man}
   \newpage
   \input{yapp.man}
%}

\input admin.ind
\end {document}
